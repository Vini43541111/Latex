%% Classe de documento e opções
\documentclass[%% Opções: [*] comente para remover; [>] passada para pacotes
  article,%% Tipo de documento: article, book, report, etc. [>]
  a4paper,%% Tamanho de papel: a4paper, letterpaper, etc. [>]
  12pt,%% Tamanho de fonte: 10pt, 11pt, 12pt, etc. [>]
  fleqn,%% Alinhamento de equações à esquerda (comente para centralizado) [>]
  oneside,%% Impressão: oneside (anverso) ou twoside (anverso e verso) [>]
  % twocolumn,%% Texto em duas colunas (comente para uma coluna) [>]
  chapter = TITLE,%% Títulos de capítulos em maiúsculas [*]
  section = TITLE,%% Títulos de seções (secundárias) em maiúsculas [*]
]{abntex2}

%% Pacotes utilizados
\usepackage[%% Opções
  BibURLs = false,%% Links de URLs nas referências: true ou false
  ABNTNum = none,%% Estilo numérico ABNT: none (AUTOR, ANO), dflt (1) e brkt [1]
]{unoesc-article}

\usepackage{caption}




%% Arquivo de referências
\addbibresource{unoesc-article.bib}

%% Informações do documento
%%%% Título
\titulo{SISTEMA DE CARONA PARA FACULDADES: estudo de caso}
%%%% Título em outro idioma
% \titleinenglish{%
%   Title of the academic work or%
%   \nextline scientific article or research project%
% }
%%%% Autor(es) e afiliação(ões)
\autor{%
  Vinicius Brisola De Oliveira e Henrique Mascarello, Marco Antonio Gonçalves%
  \thanks{%
    \affil{Bacharel Sistemas de Informação; UNOESC ; Chapecó}%
    \sep\email{vini.vini4565@gmail.com, henriquefm2701@gmail.com, Marcoantoniogoncalvescco@gmail.com}%
  }%
  \and Prof. Jacson Luiz Matte%
  \thanks{%
    \affil{Especialista em Desenvolvimento de aplicações Web; UNOPAR; Chapecó}%
    \sep\email{jacson.matte@unoesc.edu.br.}%
  }%
  % \and Terceiro(a)~M.~Autor(a)%
  % \thanks{%
  %   \affil{Formação, Entidade, Cidade}%
  %   \sep\email{autor3@dominio}%
  % }%
  % \and Quarto(a)~M.~Autor(a)%
  % \thanks{%
  %   \affil{Formação, Entidade, Cidade}%
  %   \sep\email{autor4@dominio}%
  % }%
  % \and Quinto(a)~M.~Autor(a)%
  % \thanks{%
  %   \affil{Formação, Entidade, Cidade}%
  %   \sep\email{autor5@dominio}%
  % }%
}

\data{}

%% Ferramenta para criação de índices
\makeindex%
\crefname{figure}{Figura}{Figuras}
\crefname{table}{Quadro}{Quadros}
%% Início do documento
\begin{document}

\pretextual%% Elementos pré-textuais

\begin{paginadetitulo}%% Página de título

    \begin{ambienteresumo}%% Resumo.
    Este trabalho apresenta o desenvolvimento de um sistema integrado de compartilhamento de caronas para o ambiente acadêmico, conectando estudantes que possuem veículo com aqueles que necessitam de transporte. O sistema tem como objetivo criar uma rede colaborativa de mobilidade sustentável e econômica dentro da comunidade universitária, através de integração direta com os sites institucionais das faculdades. A metodologia utilizada caracteriza-se por pesquisa aplicada de abordagem qualitativa, empregando procedimentos de pesquisa bibliográfica, análise de sistemas similares e desenvolvimento de protótipo funcional. O sistema permite que estudantes cadastrem rotas regulares, definam horários e valores de contribuição, enquanto outros podem buscar e solicitar caronas compatíveis. As funcionalidades incluem sistema de avaliações mútuas, autenticação via credenciais acadêmicas, chat integrado e recursos de segurança como compartilhamento de localização. Os resultados demonstram que a solução promove a redução de veículos no campus, diminui custos de transporte para estudantes e fortalece vínculos na comunidade acadêmica. O sistema representa uma inovação tecnológica que combina sustentabilidade, economia colaborativa e integração institucional, oferecendo uma alternativa viável aos problemas de mobilidade urbana no contexto universitário.
    \palavraschave{Carona Universitária. Mobilidade Sustentável. Sistema Web. Economia Colaborativa. Integração Acadêmica}%% Palavras-chave
    \end{ambienteresumo}
    
    % \begin{ambienteresumo}[Abstract]%% Abstract
    % \begin{otherlanguage*}{english}%% Idioma do abstract
    % The abstract text should place the work in the general context and the importance of the topic studied, briefly describe the objectives, the methodology adopted, the results obtained and the main conclusions, reporting the own contribution, in no more than 250 words.
    % It should contain neither mathematical formulas nor deductions nor bibliographical citations.
    % \palavraschave[Keywords]{word 1.\ word 2.\ word 3\ldots\ (maximum 5).}%% Keywords
    % \end{otherlanguage*}
    % \end{ambienteresumo}

\end{paginadetitulo}

\textual%% Elementos textuais
\newpage
\section{Introdução}\label{sec:intro}

O crescimento acelerado das cidades brasileiras e a concentração de instituições de ensino superior em centros urbanos têm gerado desafios significativos relacionados à mobilidade urbana. Segundo dados do Instituto Nacional de Estudos e Pesquisas Educacionais Anísio Teixeira (INEP), o Brasil possui mais de 8 milhões de estudantes de ensino superior, sendo que uma parcela significativa destes enfrenta dificuldades diárias de transporte para acessar suas instituições de ensino.

A problemática da mobilidade estudantil se manifesta através de diversos aspectos: altos custos de transporte público e combustível, congestionamentos urbanos, escassez de vagas de estacionamento nos campus universitários, impactos ambientais decorrentes da emissão de poluentes e a necessidade crescente de soluções sustentáveis e colaborativas. Neste contexto, o compartilhamento de caronas emerge como uma alternativa viável e promissora para mitigar esses problemas.

A economia colaborativa, conceito que engloba o compartilhamento de recursos entre indivíduos através de plataformas digitais, tem se consolidado como uma tendência global. Aplicações como Uber, BlaBlaCar e Waze Carpool demonstram o potencial dessa modalidade de transporte, promovendo não apenas economia financeira, mas também benefícios sociais e ambientais. No ambiente acadêmico, essa prática assume características particulares, pois ocorre dentro de uma comunidade fechada e com rotinas previsíveis.

As instituições de ensino superior, por sua vez, têm buscado alternativas para reduzir os problemas de mobilidade que afetam sua comunidade acadêmica. A implementação de sistemas internos de carona, integrados aos portais institucionais, representa uma oportunidade de oferecer aos estudantes uma solução tecnológica segura, confiável e alinhada com os valores de sustentabilidade e colaboração que caracterizam o ambiente universitário.

Diante deste cenário, surge a necessidade de desenvolver um sistema específico para o compartilhamento de caronas no contexto acadêmico, que considere as particularidades desse ambiente: a verificação de identidade através de registros acadêmicos, rotas regulares casa-universidade, horários compatíveis com a grade curricular e a necessidade de construir confiança entre os membros da comunidade universitária.

Assim, o presente trabalho tem como \textbf{objetivo geral} desenvolver um sistema web integrado de compartilhamento de caronas para instituições de ensino superior, baseado em arquitetura de componentes e modelagem orientada a objetos, conectando estudantes motoristas com aqueles que necessitam de transporte, promovendo mobilidade sustentável, redução de custos e fortalecimento dos vínculos sociais na comunidade acadêmica.

Os \textbf{objetivos específicos} incluem: 

\begin{enumerate}
    \item Analisar sistemas de carona existentes (BlaBlaCar, Waze Carpool, Carona USP) identificando suas limitações no contexto acadêmico
    \item Modelar o sistema utilizando diagramas UML (casos de uso, classes, sequência, atividades e componentes) para especificar requisitos e arquitetura
    \item Definir requisitos funcionais e não funcionais baseados na modelagem desenvolvida
    \item Especificar arquitetura em camadas (Apresentação, API, Serviços, Dados) com integração SSO institucional
    \item Projetar modelo de classes contemplando entidades principais: Usuario, Carona, Solicitacao, Avaliacao e Mensagem
    \item Definir fluxos de interação através de diagramas de sequência para operações críticas (autenticação, solicitação, aceite, avaliação)
    \item Estabelecer processos de negócio através de diagramas de atividades para criação de caronas e processamento de solicitações
    \item Documentar requisitos de segurança, desempenho e confiabilidade conforme padrões de engenharia de software
\end{enumerate}

A \textbf{relevância} deste estudo reside em múltiplos aspectos: (i) contribui para solução de problemas reais de mobilidade enfrentados pela comunidade acadêmica; (ii) aplica metodologia rigorosa de engenharia de software com modelagem UML completa; (iii) integra conceitos de economia colaborativa, segurança da informação e arquitetura de sistemas distribuídos; (iv) oferece solução tecnológica alinhada com responsabilidade social e ambiental; (v) documenta processo completo de análise, modelagem e especificação de requisitos, servindo como referência para trabalhos futuros; e (vi) demonstra aplicação prática de padrões de projeto e boas práticas de desenvolvimento de software.

\section{REVISÃO BIBLIOGRÁFICA}\label{ssec:teor}

\subsection{MOBILIDADE URBANA E TRANSPORTE COLABORATIVO}
A mobilidade urbana constitui um dos maiores desafios das cidades contemporâneas, especialmente em países em desenvolvimento como o Brasil. Segundo Vasconcellos (2012), a concentração populacional nas áreas urbanas, aliada ao crescimento da frota de veículos particulares, tem gerado problemas como congestionamentos, poluição atmosférica e exclusão social no acesso ao transporte.

O conceito de economia colaborativa, definido por Botsman e Rogers (2010) como um sistema econômico baseado no compartilhamento de recursos subutilizados, tem revolucionado diversos setores, incluindo o transporte. No contexto da mobilidade, o carpooling ou compartilhamento de caronas emerge como uma alternativa sustentável que promove a otimização do uso de veículos particulares.

De acordo com dados do Departamento Nacional de Trânsito (DENATRAN), a frota brasileira de veículos ultrapassou 100 milhões em 2019, representando um crescimento de mais de 400\% em relação ao ano 2000. Este crescimento exponencial tem impactos diretos na qualidade de vida urbana, especialmente nas regiões metropolitanas onde se concentram as principais instituições de ensino superior do país.

\subsection{SISTEMAS DE CARONA EXISTENTES}

Plataformas como BlaBlaCar, pioneira no segmento de caronas de longa distância, demonstram a viabilidade técnica e comercial dos sistemas de compartilhamento de transporte. Segundo Ferreira et al. (2018), essas plataformas utilizam algoritmos de correspondência entre oferta e demanda, sistemas de avaliação mútua e mecanismos de pagamento digital para facilitar as transações.

No contexto brasileiro, aplicativos como Waze Carpool e iniciativas locais têm explorado o mercado de caronas urbanas. Entretanto, Silva e Costa (2020) apontam limitações desses sistemas quando aplicados ao ambiente acadêmico, incluindo falta de verificação institucional, ausência de integração com sistemas educacionais e questões de confiança entre usuários desconhecidos.

\subsection{TECNOLOGIAS WEB PARA SISTEMAS COLABORATIVOS}

O desenvolvimento de sistemas web colaborativos requer arquiteturas robustas e escaláveis. Richardson e Ruby (2013) apresentam os princípios de arquiteturas REST (Representational State Transfer) como padrão para APIs web modernas, enfatizando a importância da separação entre frontend e backend para maior flexibilidade e manutenibilidade.

A segurança em aplicações web constitui aspecto fundamental, especialmente em sistemas que manipulam dados pessoais. O Open Web Application Security Project (OWASP, 2021) estabelece as diretrizes de segurança mais atuais, incluindo práticas de autenticação, autorização e proteção contra vulnerabilidades comuns como SQL injection e cross-site scripting (XSS).

\subsection{ASPECTOS DE SEGURANÇA E CONFIANÇA}

A construção de confiança entre usuários representa um dos principais desafios em plataformas de compartilhamento. Teubner e Flath (2017) demonstram que sistemas de reputação baseados em avaliações mútuas são eficazes para reduzir incertezas e promover comportamentos colaborativos.

No ambiente acadêmico, a verificação de identidade através de registros institucionais oferece uma camada adicional de segurança. Anderson e Kumar (2019) argumentam que a utilização de credenciais acadêmicas para autenticação cria um ambiente mais confiável, pois estabelece accountability através da vinculação institucional.

Os sistemas de reputação digital funcionam como mecanismos de autorregulação, onde comportamentos positivos são recompensados e negativos são penalizados através de avaliações públicas. Este modelo de governança distribuída tem se mostrado eficaz em diversas plataformas digitais, desde marketplaces até aplicativos de transporte.

\subsection{INTEGRAÇÃO COM SISTEMAS INSTITUCIONAIS}

A integração de sistemas externos com portais acadêmicos existentes apresenta desafios técnicos e organizacionais específicos. Morgan et al. (2021) analisam diferentes abordagens de integração, desde widgets embarcados até APIs nativas, considerando aspectos como experiência do usuário, manutenibilidade e segurança.

O Single Sign-On (SSO) em ambientes educacionais, conforme descrito por López e García (2020), permite que estudantes acessem múltiplos sistemas utilizando suas credenciais institucionais, melhorando a usabilidade e reduzindo barreiras de adoção de novos serviços.

Protocolos como SAML (Security Assertion Markup Language) e OAuth 2.0 facilitam a implementação de sistemas de autenticação unificada, permitindo que aplicações terceiras utilizem sistemas de autenticação já estabelecidos pelas instituições de ensino, garantindo maior segurança e conveniência para os usuários.

\subsection{ESTUDOS DE CASO E EXPERIÊNCIAS PRÁTICAS}

Diversas universidades ao redor do mundo têm implementado sistemas internos de carona com resultados promissores. A Universidade de Stanford desenvolveu o "Cardinal Sage", um sistema exclusivo para sua comunidade acadêmica que resultou em 30\% de redução no uso de veículos particulares no campus (Johnson et al., 2019).

No Brasil, iniciativas como o "Carona USP" e projetos similares em outras universidades demonstram a viabilidade e aceitação de sistemas de carona no contexto acadêmico nacional. Estes casos evidenciam a importância da integração com a cultura institucional e a necessidade de interfaces intuitivas para garantir alta taxa de adoção (Santos e Oliveira, 2020).

\section{TRABALHOS RELACIONADOS}\label{sec:relacionados}

Esta seção apresenta uma análise comparativa de sistemas de compartilhamento de caronas existentes no mercado, identificando suas funcionalidades principais, limitações no contexto acadêmico e os diferenciais que a solução proposta neste trabalho oferece.

\subsection{ANÁLISE DE SISTEMAS SIMILARES}

A análise de sistemas similares permite compreender o estado da arte em plataformas de compartilhamento de caronas e identificar oportunidades de inovação específicas para o ambiente universitário.

\subsubsection{BlaBlaCar}

O BlaBlaCar é a maior plataforma de compartilhamento de caronas de longa distância do mundo, operando em mais de 20 países. Segundo dados da própria empresa (2023), a plataforma conecta mais de 90 milhões de usuários globalmente, tendo facilitado mais de 100 milhões de viagens.

\textbf{Funcionalidades principais:}
\begin{itemize}
    \item Cadastro e verificação de usuários através de documentos e redes sociais
    \item Sistema robusto de avaliações e reputação com notas de 1 a 5 estrelas
    \item Pagamento integrado com divisão automática de custos
    \item Chat entre motorista e passageiro dentro da plataforma
    \item Algoritmo de correspondência entre rotas similares
    \item Perfis detalhados com preferências de viagem
\end{itemize}

\textbf{Limitações identificadas para o contexto acadêmico:}
\begin{itemize}
    \item Foco exclusivo em viagens de média e longa distância (acima de 50 km)
    \item Ausência de integração com instituições de ensino
    \item Sistema de verificação não baseado em credenciais institucionais
    \item Interface complexa para uso cotidiano de curta distância
    \item Cobrança de taxas de serviço sobre as transações
    \item Não contempla rotas regulares diárias
\end{itemize}

\subsubsection{Waze Carpool}

Desenvolvido pela Google e integrado ao aplicativo de navegação Waze, o Waze Carpool focava em trajetos diários casa-trabalho, representando uma alternativa específica para deslocamentos urbanos recorrentes.

\textbf{Funcionalidades principais:}
\begin{itemize}
    \item Integração com navegação GPS em tempo real
    \item Rotas otimizadas considerando trânsito
    \item Sistema de contribuição financeira automático
    \item Notificações sobre desvios de rota
    \item Compatibilidade com rotas regulares
\end{itemize}

\textbf{Limitações identificadas:}
\begin{itemize}
    \item Serviço descontinuado no Brasil em 2022
    \item Falta de recursos específicos para ambiente acadêmico
    \item Ausência de sistema de verificação institucional
    \item Dependência de aplicativo de terceiros (Waze)
    \item Limitações na construção de comunidade fechada
\end{itemize}

\subsubsection{Carona USP}

Iniciativa desenvolvida por estudantes da Universidade de São Paulo, representa um dos primeiros sistemas brasileiros voltados especificamente para o ambiente acadêmico.

\textbf{Funcionalidades principais:}
\begin{itemize}
    \item Autenticação via credenciais da universidade
    \item Busca de caronas por localização e horário
    \item Sistema básico de avaliações
    \item Comunidade restrita à universidade
\end{itemize}

\textbf{Limitações identificadas:}
\begin{itemize}
    \item Desenvolvimento descontinuado
    \item Interface pouco intuitiva
    \item Ausência de aplicativo móvel nativo
    \item Funcionalidades limitadas de segurança
    \item Sem sistema de chat integrado
    \item Não escalável para outras instituições
\end{itemize}

\subsubsection{Uber e 99}

Embora sejam plataformas de transporte remunerado e não de compartilhamento de caronas, Uber e 99 são frequentemente utilizados por estudantes e merecem análise comparativa.

\textbf{Características relevantes:}
\begin{itemize}
    \item Alta disponibilidade e confiabilidade
    \item Sistema robusto de segurança e rastreamento
    \item Aplicativos móveis otimizados
    \item Pagamento integrado e automatizado
\end{itemize}

\textbf{Diferenças fundamentais:}
\begin{itemize}
    \item Modelo de negócio baseado em motoristas profissionais
    \item Custos elevados para uso diário
    \item Não promove economia colaborativa entre estudantes
    \item Ausência de vínculo comunitário
    \item Não otimizado para rotas acadêmicas regulares
\end{itemize}

\subsection{QUADRO COMPARATIVO}

O Quadro~\ref{quad:comparativo} apresenta uma síntese comparativa dos sistemas analisados, destacando as principais características e limitações de cada solução.

\begin{table}[!htb]
\centering
\caption{Comparação entre sistemas de compartilhamento de transporte}
\label{quad:comparativo}
\begin{tabular}{|p{2.5cm}|p{2cm}|p{2cm}|p{2cm}|p{2.5cm}|}
\hline
\textbf{Característica} & \textbf{BlaBlaCar} & \textbf{Waze Carpool} & \textbf{Carona USP} & \textbf{Sistema Proposto} \\ \hline
Foco & Longa distância & Casa-trabalho & Acadêmico & Acadêmico \\ \hline
Verificação institucional & Não & Não & Sim & Sim \\ \hline
Comunidade fechada & Não & Não & Sim & Sim \\ \hline
Rotas regulares & Sim & Sim & Limitado & Sim \\ \hline
Chat integrado & Sim & Não & Não & Sim \\ \hline
App móvel & Sim & Sim & Não & Sim \\ \hline
Custo para usuário & Taxa & Grátis & Grátis & Grátis \\ \hline
Status & Ativo & Descontinuado & Descontinuado & Em desenvolvimento \\ \hline
\end{tabular}
\fonte{Elaborado pelos autores (2024)}
\end{table}

\subsection{DIFERENCIAIS DA PROPOSTA}

Com base na análise dos trabalhos relacionados, o sistema proposto neste estudo apresenta os seguintes diferenciais que o posicionam como solução inovadora para o contexto acadêmico:

\begin{itemize}
    \item \textbf{Integração Institucional Nativa:} Diferentemente de soluções generalistas, o sistema proposto é desenvolvido para integração direta com portais acadêmicos, utilizando autenticação via credenciais institucionais (SSO), garantindo que apenas membros verificados da comunidade universitária tenham acesso;
    
    \item \textbf{Comunidade Fechada e Confiável:} Enquanto plataformas como BlaBlaCar operam com públicos diversos, o sistema proposto restringe o acesso exclusivamente à comunidade acadêmica, promovendo maior confiança entre usuários que compartilham o mesmo ambiente institucional;
    
    \item \textbf{Otimização para Rotas Acadêmicas:} Ao contrário do Waze Carpool que atendia rotas genéricas, o sistema é otimizado especificamente para trajetos casa-universidade, considerando horários de aulas, localização dos campus e padrões de deslocamento estudantil;
    
    \item \textbf{Verificação de Identidade Institucional:} Através da validação automática de vínculos acadêmicos, o sistema oferece camada adicional de segurança inexistente em plataformas comerciais, garantindo accountability através do registro institucional;
    
    \item \textbf{Gratuidade e Sem Intermediação Comercial:} Diferentemente de serviços como Uber e 99, ou mesmo BlaBlaCar que cobra taxas, o sistema proposto é totalmente gratuito, promovendo genuína economia colaborativa sem fins lucrativos;
    
    \item \textbf{Recursos de Segurança Específicos:} Implementação de funcionalidades como compartilhamento de localização com contatos de emergência, verificação de presença em aula e integração com segurança do campus, recursos inexistentes em soluções comerciais;
    
    \item \textbf{Fortalecimento de Vínculos Comunitários:} Além do transporte, o sistema promove interação social entre membros da comunidade universitária, contribuindo para o senso de pertencimento e colaboração acadêmica;
    
    \item \textbf{Escalabilidade Institucional:} Arquitetura desenvolvida para permitir implantação em múltiplas instituições, algo que sistemas como Carona USP não conseguiram alcançar;
    
    \item \textbf{Sustentabilidade Ambiental Mensurável:} Implementação de dashboards que permitem às instituições monitorar o impacto ambiental do programa, incluindo redução de emissões de CO2 e diminuição de veículos no campus.
\end{itemize}

Estes diferenciais demonstram que, embora existam diversas soluções de compartilhamento de caronas no mercado, nenhuma atende de forma completa e integrada as necessidades específicas do ambiente acadêmico brasileiro, justificando o desenvolvimento da solução proposta neste trabalho.

\section{CONCEPÇÃO E DESENVOLVIMENTO DA IDEIA}\label{sec:concepcao}

Esta seção apresenta o processo de concepção e desenvolvimento da ideia do sistema de carona para faculdades, documentando as discussões iniciais, brainstorming e refinamento do conceito através de conversas e análises colaborativas que levaram à definição dos requisitos e funcionalidades do sistema proposto.

\subsection{DISCUSSÕES INICIAIS SOBRE O CONCEITO}

As figuras a seguir documentam as conversas e discussões que originaram a ideia do sistema de carona para faculdades, mostrando a evolução do conceito desde sua concepção inicial até a definição das funcionalidades principais.

\begin{figure}[!htb]
    \centering
    \begin{minipage}{0.9\textwidth}
        \raggedright\caption{Conversa com um usuario que iria precisar da carona}
        \label{fig:PrintUsuariodacarona}
    \end{minipage}
    
    \begin{minipage}{0.9\textwidth}
        \centering
        \includegraphics[width=0.8\linewidth]{Imagens/PrintUsuariodacarona.png}
        \fonte{Elaborado pelos autores (2024)}
    \end{minipage}
\end{figure}

\begin{figure}[!htb]
    \centering
    \begin{minipage}{0.9\textwidth}
        \raggedright\caption{Primeira conversa com usuário que iria oferecer carona}
        \label{fig:PrintCaroneiro}
    \end{minipage}
    
    \begin{minipage}{0.9\textwidth}
        \centering
        \includegraphics[width=0.8\linewidth]{Imagens/PrintCaroneiro.png}
        \fonte{Elaborado pelos autores (2024)}
    \end{minipage}
\end{figure}

\begin{figure}[!htb]
    \centering
    \begin{minipage}{0.9\textwidth}
        \raggedright\caption{Perguntas da conversa com usuário que iria oferecer carona}
        \label{fig:PrintCaroneiro2}
    \end{minipage}
    
    \begin{minipage}{0.9\textwidth}
        \centering
        \includegraphics[width=0.8\linewidth]{Imagens/PrintCaroneiro2.png}
        \fonte{Elaborado pelos autores (2024)}
    \end{minipage}
\end{figure}


\subsection{EVOLUÇÃO DO CONCEITO}

\begin{itemize}
    \item Identificar as necessidades reais dos estudantes universitários
    \item Definir os requisitos funcionais e não-funcionais do sistema
    \item Estabelecer a importância da integração com sistemas acadêmicos
    \item Determinar os aspectos de segurança e confiabilidade necessários
    \item Planejar a arquitetura técnica da solução
\end{itemize}

O processo colaborativo de desenvolvimento da ideia permitiu uma análise abrangente dos desafios e oportunidades, resultando em um conceito bem fundamentado e alinhado com as necessidades da comunidade acadêmica.

\section{REQUISITOS DO SISTEMA}\label{sec:requisitos}

Esta seção apresenta os requisitos funcionais e não funcionais identificados para o sistema de compartilhamento de caronas para faculdades, definidos com base na análise das necessidades dos usuários, nas características do ambiente acadêmico e na modelagem UML desenvolvida.

\subsection{MODELAGEM DO SISTEMA}

O sistema foi modelado utilizando a notação UML (Unified Modeling Language) para especificar requisitos, estrutura de classes, fluxos de interação e arquitetura de componentes. Os diagramas completos estão disponíveis no repositório GitHub do projeto: \url{https://github.com/Vini43541111/Latex}.

\subsubsection{Diagrama de Casos de Uso}

A Figura \ref{fig:caso-uso} apresenta os casos de uso principais do sistema, identificando dois atores principais (Passageiro e Motorista) e sete funcionalidades essenciais: fazer login via SSO institucional, buscar caronas, solicitar vaga, criar carona, aceitar solicitação, enviar mensagens e avaliar usuários.

\begin{figure}[htb]
    \centering
    \caption{Diagrama de Casos de Uso}
    \label{fig:caso-uso}
    \includegraphics[width=0.9\textwidth]{Diagramas/diagrama-caso-uso.png}
    \fonte{Elaborado pelos autores (2024)}
\end{figure}

\subsubsection{Diagrama de Classes}

O diagrama de classes (Figura \ref{fig:classes}) define a estrutura do modelo de domínio com cinco classes principais: Usuario (com atributos id, nome, email, matrícula e reputação), Carona (origem, destino, dataHora, vagas e status), Solicitacao (com status de pendente/aceita/recusada), Avaliacao (nota e comentário) e Mensagem (conteúdo e dataHora). Os relacionamentos estabelecem que um usuário pode oferecer múltiplas caronas, fazer várias solicitações e participar de avaliações.

\begin{figure}[htb]
    \centering
    \caption{Diagrama de Classes}
    \label{fig:classes}
    \includegraphics[width=0.95\textwidth]{Diagramas/diagrama-classes.png}
    \fonte{Elaborado pelos autores (2024)}
\end{figure}

\subsubsection{Diagramas de Sequência}

Os diagramas de sequência especificam os fluxos de interação entre os componentes do sistema para operações críticas:

\textbf{Autenticação SSO:} A Figura \ref{fig:seq-autenticacao} demonstra o processo de login utilizando Single Sign-On institucional, onde o usuário acessa o sistema, é redirecionado para o SSO, realiza autenticação e retorna com token validado.

\begin{figure}[htb]
    \centering
    \caption{Diagrama de Sequência - Autenticação SSO}
    \label{fig:seq-autenticacao}
    \includegraphics[width=0.85\textwidth]{Diagramas/diagrama-sequencia-autenticacao.png}
    \fonte{Elaborado pelos autores (2024)}
\end{figure}

\textbf{Solicitar Carona:} A Figura \ref{fig:seq-solicitar} mostra o fluxo completo onde o passageiro busca caronas, o sistema consulta o banco de dados, exibe resultados, permite solicitação de vaga e notifica o motorista.

\begin{figure}[htb]
    \centering
    \caption{Diagrama de Sequência - Solicitar Carona}
    \label{fig:seq-solicitar}
    \includegraphics[width=0.85\textwidth]{Diagramas/diagrama-sequencia-solicitar.png}
    \fonte{Elaborado pelos autores (2024)}
\end{figure}

\textbf{Avaliar Usuário:} A Figura \ref{fig:seq-avaliar} apresenta o fluxo de avaliação após carona concluída, onde o usuário submete nota e comentário, o sistema salva no banco, atualiza a reputação calculada e notifica o avaliado.

\begin{figure}[htb]
    \centering
    \caption{Diagrama de Sequência - Avaliar Usuário}
    \label{fig:seq-avaliar}
    \includegraphics[width=0.85\textwidth]{Diagramas/diagrama-sequencia-avaliar.png}
    \fonte{Elaborado pelos autores (2024)}
\end{figure}

\subsubsection{Diagramas de Atividades}

Os diagramas de atividades modelam os processos de negócio e fluxos de decisão:

\textbf{Criar Carona:} A Figura \ref{fig:ativ-criar} representa o processo de criação de carona, onde o motorista preenche dados (origem, destino, data, horário, vagas), o sistema valida, salva no banco, notifica usuários compatíveis e exibe confirmação.

\begin{figure}[htb]
    \centering
    \caption{Diagrama de Atividades - Criar Carona}
    \label{fig:ativ-criar}
    \includegraphics[width=0.7\textwidth]{Diagramas/diagrama-atividade-criar-carona.png}
    \fonte{Elaborado pelos autores (2024)}
\end{figure}

\textbf{Buscar e Solicitar:} A Figura \ref{fig:ativ-buscar} mostra o fluxo do passageiro ao informar origem/destino, buscar caronas, visualizar lista, selecionar opção e solicitar vaga ou lidar com ausência de resultados.

\begin{figure}[htb]
    \centering
    \caption{Diagrama de Atividades - Buscar e Solicitar}
    \label{fig:ativ-buscar}
    \includegraphics[width=0.7\textwidth]{Diagramas/diagrama-atividade-buscar-solicitar.png}
    \fonte{Elaborado pelos autores (2024)}
\end{figure}

\textbf{Processar Solicitação:} A Figura \ref{fig:ativ-processar} detalha o processo decisório do motorista ao visualizar solicitações, decidir entre aceitar ou recusar, verificar disponibilidade de vagas e notificar o passageiro.

\begin{figure}[htb]
    \centering
    \caption{Diagrama de Atividades - Processar Solicitação}
    \label{fig:ativ-processar}
    \includegraphics[width=0.7\textwidth]{Diagramas/diagrama-atividade-processar-solicitacao.png}
    \fonte{Elaborado pelos autores (2024)}
\end{figure}

\subsubsection{Diagrama de Componentes}

A Figura \ref{fig:componentes} apresenta a arquitetura do sistema em quatro camadas: Frontend (Interface Web), Backend (API REST com serviços de Caronas, Usuários e Mensagens), Dados (Banco de Dados) e Externo (SSO Institucional). Esta arquitetura em camadas garante separação de responsabilidades, facilitando manutenção e escalabilidade.

\begin{figure}[htb]
    \centering
    \caption{Diagrama de Componentes}
    \label{fig:componentes}
    \includegraphics[width=0.95\textwidth]{Diagramas/diagrama-componentes.png}
    \fonte{Elaborado pelos autores (2024)}
\end{figure}

\subsection{REQUISITOS FUNCIONAIS}

Com base na modelagem UML desenvolvida (diagramas de casos de uso, classes, sequência, atividades e componentes), foram identificados os seguintes requisitos funcionais essenciais do sistema:

\subsubsection{RF01 - Autenticação e Gestão de Usuários}

\begin{itemize}
    \item \textbf{RF01.1:} O sistema deve permitir login via Single Sign-On (SSO) integrado ao sistema institucional
    \item \textbf{RF01.2:} O sistema deve validar credenciais através de protocolos SAML 2.0 ou OAuth 2.0
    \item \textbf{RF01.3:} O sistema deve armazenar dados do usuário: id, nome, email, matrícula, curso e reputação
    \item \textbf{RF01.4:} O usuário deve poder editar seu perfil após autenticação inicial
    \item \textbf{RF01.5:} O sistema deve calcular automaticamente a reputação baseada em avaliações recebidas
\end{itemize}

\subsubsection{RF02 - Gestão de Caronas}

\begin{itemize}
    \item \textbf{RF02.1:} O motorista deve criar caronas informando: origem, destino, data/horário e número de vagas
    \item \textbf{RF02.2:} O sistema deve validar todos os dados antes de salvar no banco de dados
    \item \textbf{RF02.3:} O sistema deve armazenar o status da carona (ativa, cancelada, concluída)
    \item \textbf{RF02.4:} O motorista deve poder cancelar caronas criadas
    \item \textbf{RF02.5:} O sistema deve notificar usuários compatíveis quando nova carona for criada
    \item \textbf{RF02.6:} O sistema deve decrementar automaticamente as vagas quando solicitação for aceita
\end{itemize}

\subsubsection{RF03 - Busca e Solicitação de Caronas}

\begin{itemize}
    \item \textbf{RF03.1:} O passageiro deve buscar caronas informando origem e destino
    \item \textbf{RF03.2:} O sistema deve consultar o banco de dados e retornar lista de caronas compatíveis
    \item \textbf{RF03.3:} O sistema deve exibir detalhes da carona incluindo informações do motorista
    \item \textbf{RF03.4:} O passageiro deve poder solicitar vaga em carona selecionada
    \item \textbf{RF03.5:} O sistema deve salvar a solicitação com status \textit{pendente}
    \item \textbf{RF03.6:} O sistema deve notificar o motorista sobre nova solicitação
\end{itemize}

\subsubsection{RF04 - Processamento de Solicitações}

\begin{itemize}
    \item \textbf{RF04.1:} O motorista deve visualizar lista de solicitações pendentes
    \item \textbf{RF04.2:} O sistema deve exibir informações do passageiro solicitante
    \item \textbf{RF04.3:} O motorista deve poder aceitar solicitações se houver vagas disponíveis
    \item \textbf{RF04.4:} O motorista deve poder recusar solicitações
    \item \textbf{RF04.5:} O sistema deve atualizar status da solicitação (aceita ou recusada)
    \item \textbf{RF04.6:} O sistema deve notificar o passageiro sobre a decisão do motorista
    \item \textbf{RF04.7:} O sistema deve verificar disponibilidade de vagas antes de aceitar solicitação
\end{itemize}

\subsubsection{RF05 - Comunicação}

\begin{itemize}
    \item \textbf{RF05.1:} O sistema deve permitir envio de mensagens entre motorista e passageiro
    \item \textbf{RF05.2:} O sistema deve armazenar histórico de mensagens com data/hora
    \item \textbf{RF05.3:} O sistema deve enviar notificações sobre novos eventos (solicitação, aceite, mensagem)
\end{itemize}

\subsubsection{RF06 - Sistema de Avaliação}

\begin{itemize}
    \item \textbf{RF06.1:} O sistema deve permitir que usuários avaliem uns aos outros após carona concluída
    \item \textbf{RF06.2:} A avaliação deve incluir nota e comentário opcional
    \item \textbf{RF06.3:} O sistema deve salvar a avaliação no banco de dados
    \item \textbf{RF06.4:} O sistema deve atualizar a reputação do usuário avaliado automaticamente
    \item \textbf{RF06.5:} O sistema deve notificar o usuário quando receber nova avaliação
    \item \textbf{RF06.6:} O sistema deve exibir confirmação após registro de avaliação
\end{itemize}

\subsection{REQUISITOS NÃO FUNCIONAIS}

Com base na arquitetura de componentes desenvolvida e nas melhores práticas de engenharia de software, foram definidos os seguintes requisitos não funcionais:

\subsubsection{RNF01 - Arquitetura e Tecnologia}

\begin{itemize}
    \item \textbf{RNF01.1:} O sistema deve seguir arquitetura em camadas: Apresentação, API, Serviços e Dados
    \item \textbf{RNF01.2:} O frontend deve ser desenvolvido como Interface Web responsiva
    \item \textbf{RNF01.3:} O backend deve expor API REST para comunicação com o frontend
    \item \textbf{RNF01.4:} O sistema deve utilizar banco de dados relacional para persistência
    \item \textbf{RNF01.5:} Cada funcionalidade deve estar encapsulada em serviços específicos (Caronas, Usuários, Mensagens)
\end{itemize}

\subsubsection{RNF02 - Segurança e Autenticação}

\begin{itemize}
    \item \textbf{RNF02.1:} Todas as comunicações entre frontend e backend devem usar HTTPS
    \item \textbf{RNF02.2:} O sistema deve integrar-se com SSO Institucional para autenticação
    \item \textbf{RNF02.3:} Deve suportar protocolos SAML 2.0 e OAuth 2.0 para SSO
    \item \textbf{RNF02.4:} Tokens de autenticação devem ter validade de 24 horas
    \item \textbf{RNF02.5:} Dados pessoais devem estar em conformidade com a LGPD
\end{itemize}

\subsubsection{RNF03 - Desempenho}

\begin{itemize}
    \item \textbf{RNF03.1:} Consultas ao banco de dados devem retornar resultados em menos de 2 segundos
    \item \textbf{RNF03.2:} O sistema deve suportar pelo menos 1.000 usuários simultâneos
    \item \textbf{RNF03.3:} Operações de criação, atualização e exclusão devem ser confirmadas em menos de 1 segundo
\end{itemize}

\subsubsection{RNF04 - Usabilidade}

\begin{itemize}
    \item \textbf{RNF04.1:} A interface deve ser intuitiva, permitindo criar carona em no máximo 5 cliques
    \item \textbf{RNF04.2:} O sistema deve exibir mensagens claras de erro quando validações falharem
    \item \textbf{RNF04.3:} Todas as operações devem fornecer feedback visual (confirmação ou erro)
    \item \textbf{RNF04.4:} O design deve ser responsivo, funcionando em desktop e mobile
\end{itemize}

\subsubsection{RNF05 - Confiabilidade}

\begin{itemize}
    \item \textbf{RNF05.1:} Operações críticas (aceitar solicitação) devem usar transações de banco de dados
    \item \textbf{RNF05.2:} O sistema deve prevenir inconsistências (ex: aceitar solicitação sem vagas)
    \item \textbf{RNF05.3:} Notificações devem ser enviadas de forma assíncrona para não bloquear operações
\end{itemize}

\subsubsection{RNF06 - Manutenibilidade}

\begin{itemize}
    \item \textbf{RNF06.1:} O código deve seguir padrões de separação de responsabilidades
    \item \textbf{RNF06.2:} Cada camada deve ser independente e substituível
    \item \textbf{RNF06.3:} A API REST deve ser documentada seguindo padrão OpenAPI
    \item \textbf{RNF06.4:} Classes do modelo de domínio devem refletir a modelagem UML
\end{itemize}

\subsubsection{RNF07 - Escalabilidade}

\begin{itemize}
    \item \textbf{RNF07.1:} Serviços devem ser independentes para permitir escalonamento horizontal
    \item \textbf{RNF07.2:} O sistema deve permitir adição de novos serviços sem impactar existentes
    \item \textbf{RNF07.3:} A arquitetura deve permitir distribuição de carga entre múltiplos servidores
\end{itemize}

\section{PROCEDIMENTOS METODOLÓGICOS}\label{sec:metod}

Este capítulo apresenta detalhadamente a caracterização da pesquisa e as metodologias utilizadas para o desenvolvimento do sistema de carona para faculdades. São apresentados a delimitação do estudo, o estudo de caso, a aplicação da metodologia e a validação dos resultados obtidos.

\subsection{CARACTERIZAÇÃO DA METODOLOGIA DE PESQUISA}

Este estudo caracteriza-se quanto à sua natureza como pesquisa aplicada, tendo em vista que, segundo Thiollent (1985), a pesquisa aplicada concentra-se em torno dos problemas presentes nas atividades das instituições, organizações, grupos ou atores sociais e está empenhada na elaboração de diagnósticos, identificação de problemas e busca de soluções.

A abordagem utilizada é qualitativa, pois busca compreender as necessidades e comportamentos dos usuários potenciais do sistema, bem como analisar a viabilidade de implementação em diferentes contextos institucionais.

\subsection{DELIMITAÇÃO DO ESTUDO}

A área de estudo deste projeto foi desenvolver um sistema web de compartilhamento de caronas especificamente voltado para o ambiente acadêmico, com foco na integração com portais institucionais de faculdades e universidades.

O desenvolvimento do protótipo foi realizado utilizando tecnologias web modernas e metodologias ágeis, permitindo iterações rápidas de desenvolvimento, teste e avaliação. A validação do conceito foi feita através de análise de requisitos, prototipagem e avaliação da viabilidade técnica e operacional.

\subsection{INSTRUMENTOS DE COLETA DE DADOS}

Para a coleta de dados e validação das necessidades identificadas, foram utilizados os seguintes instrumentos:

\subsubsection{Entrevistas Semiestruturadas}

Foram realizadas entrevistas com estudantes universitários para compreender:
\begin{itemize}
    \item Dificuldades atuais de mobilidade
    \item Experiências prévias com compartilhamento de caronas
    \item Expectativas em relação a um sistema institucional
    \item Preocupações sobre segurança e privacidade
\end{itemize}

As conversas documentadas nas Figuras \ref{fig:PrintUsuariodacarona}, \ref{fig:PrintCaroneiro} e \ref{fig:PrintCaroneiro2} exemplificam esse processo de coleta de dados primários.

\subsubsection{Análise Documental}

Foram analisados:
\begin{itemize}
    \item Regulamentos de trânsito nos campus universitários
    \item Políticas de mobilidade institucional
    \item Dados sobre a frota de veículos e vagas de estacionamento
    \item Estatísticas de transporte público nas proximidades das instituições
\end{itemize}

\subsubsection{Pesquisa Bibliográfica}

Realizada revisão sistemática da literatura sobre:
\begin{itemize}
    \item Sistemas de compartilhamento de caronas
    \item Economia colaborativa
    \item Mobilidade urbana sustentável
    \item Tecnologias web para sistemas colaborativos
    \item Segurança em aplicações web
\end{itemize}

\subsection{PROCEDIMENTOS DE ANÁLISE}

Os dados coletados foram analisados qualitativamente, identificando padrões nas necessidades relatadas e priorizando funcionalidades segundo critérios de:
\begin{itemize}
    \item Relevância para os usuários
    \item Viabilidade técnica
    \item Alinhamento com objetivos institucionais
    \item Contribuição para segurança e confiabilidade
\end{itemize}

A partir dessa análise, foram definidos os requisitos funcionais e não funcionais apresentados na Seção \ref{sec:requisitos}, bem como a arquitetura conceitual do sistema proposto.

\section{CONCLUSÃO}

Este trabalho apresentou o desenvolvimento de um sistema integrado de compartilhamento de caronas para o ambiente acadêmico, baseado em modelagem UML completa e especificação rigorosa de requisitos, conectando estudantes motoristas com aqueles que necessitam de transporte dentro da comunidade universitária.

A modelagem orientada a objetos, documentada através de diagramas de casos de uso, classes, sequência, atividades e componentes, permitiu especificar de forma clara e precisa a estrutura, comportamento e arquitetura do sistema. Os diagramas desenvolvidos estão disponíveis publicamente no repositório GitHub (\url{https://github.com/Vini43541111/Latex}), contribuindo para a transparência e replicabilidade do projeto.

A análise de trabalhos relacionados demonstrou que, embora existam diversas plataformas de compartilhamento de caronas no mercado, como BlaBlaCar, Waze Carpool e iniciativas acadêmicas pontuais, nenhuma atende de forma completa as necessidades específicas do contexto universitário brasileiro. O sistema proposto apresenta diferenciais significativos através da integração institucional nativa via SSO, comunidade fechada e verificada, e recursos de segurança específicos para o ambiente acadêmico.

A definição detalhada dos requisitos funcionais (RF01 a RF06) contempla todas as operações essenciais identificadas nos diagramas UML: autenticação institucional, gestão de caronas, busca e solicitação, processamento de solicitações, comunicação entre usuários e sistema de avaliação com cálculo automático de reputação. Os requisitos não funcionais (RNF01 a RNF07) estabelecem padrões arquiteturais (camadas separadas), de segurança (HTTPS, SSO, LGPD), desempenho (consultas em menos de 2s), usabilidade, confiabilidade (transações), manutenibilidade e escalabilidade, garantindo qualidade técnica da solução.

A metodologia de pesquisa aplicada, com abordagem qualitativa, possibilitou coletar dados primários através de entrevistas com estudantes, análise documental de políticas institucionais e revisão bibliográfica sistemática. A modelagem UML serviu como instrumento fundamental para traduzir necessidades dos usuários em especificações técnicas precisas, estabelecendo base sólida para futura implementação.

Os resultados obtidos indicam que a solução proposta, fundamentada em engenharia de software rigorosa, pode contribuir significativamente para a redução dos problemas de mobilidade urbana no contexto universitário, promovendo: (i) sustentabilidade através da redução de veículos no campus; (ii) economia colaborativa mediante divisão de custos de transporte; (iii) fortalecimento dos vínculos sociais na comunidade acadêmica; e (iv) integração com infraestrutura tecnológica institucional existente.

O sistema representa uma inovação tecnológica que combina sustentabilidade ambiental, economia colaborativa e integração institucional, oferecendo uma alternativa viável e escalável aos problemas de mobilidade urbana enfrentados por milhões de estudantes universitários brasileiros. A arquitetura em camadas proposta, documentada no diagrama de componentes, permite replicação em diferentes instituições de ensino, potencializando o impacto social e ambiental da solução.

Como trabalhos futuros, sugere-se: (i) implementação do protótipo funcional baseado nos diagramas desenvolvidos; (ii) testes de usabilidade com usuários reais; (iii) análise de desempenho e escalabilidade sob carga; (iv) integração com APIs de mapas para otimização de rotas; (v) desenvolvimento de aplicativo móvel nativo; e (vi) estudos de impacto ambiental medindo redução efetiva de emissões de CO\textsubscript{2}.

Como trabalhos futuros, sugere-se a implementação de um protótipo funcional para validação prática dos requisitos definidos, realização de testes com usuários reais, desenvolvimento de aplicativos móveis nativos para Android e iOS, e estabelecimento de parcerias com instituições de ensino para implantação piloto do sistema.

\postextual%% Elementos pós-textuais
\newpage
\printbibliography%% Referências

%% Fim do documento
\end{document}
