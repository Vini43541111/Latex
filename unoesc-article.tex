%% Classe de documento e opções
\documentclass[%% Opções: [*] comente para remover; [>] passada para pacotes
  article,%% Tipo de documento: article, book, report, etc. [>]
  a4paper,%% Tamanho de papel: a4paper, letterpaper, etc. [>]
  12pt,%% Tamanho de fonte: 10pt, 11pt, 12pt, etc. [>]
  fleqn,%% Alinhamento de equações à esquerda (comente para centralizado) [>]
  oneside,%% Impressão: oneside (anverso) ou twoside (anverso e verso) [>]
  % twocolumn,%% Texto em duas colunas (comente para uma coluna) [>]
  chapter = TITLE,%% Títulos de capítulos em maiúsculas [*]
  section = TITLE,%% Títulos de seções (secundárias) em maiúsculas [*]
]{abntex2}

%% Pacotes utilizados
\usepackage[%% Opções
  BibURLs = false,%% Links de URLs nas referências: true ou false
  ABNTNum = none,%% Estilo numérico ABNT: none (AUTOR, ANO), dflt (1) e brkt [1]
]{unoesc-article}

\usepackage{caption}




%% Arquivo de referências
\addbibresource{unoesc-article.bib}

%% Informações do documento
%%%% Título
\titulo{SISTEMA DE CARONA PARA FACULDADES: estudo de caso}
%%%% Título em outro idioma
% \titleinenglish{%
%   Title of the academic work or%
%   \nextline scientific article or research project%
% }
%%%% Autor(es) e afiliação(ões)
\autor{%
  Vinicius Brisola De Oliveira e Henrique Mascarello%
  \thanks{%
    \affil{Bacharel Sistemas de Informação; UNOESC ; Chapecó}%
    \sep\email{vini.vini4565@gmail.com, henriquefm2701@gmail.com}%
  }%
  \and Prof. Jacson Luiz Matte%
  \thanks{%
    \affil{Especialista em Desenvolvimento de aplicações Web; UNOPAR; Chapecó}%
    \sep\email{jacson.matte@unoesc.edu.br.}%
  }%
  % \and Terceiro(a)~M.~Autor(a)%
  % \thanks{%
  %   \affil{Formação, Entidade, Cidade}%
  %   \sep\email{autor3@dominio}%
  % }%
  % \and Quarto(a)~M.~Autor(a)%
  % \thanks{%
  %   \affil{Formação, Entidade, Cidade}%
  %   \sep\email{autor4@dominio}%
  % }%
  % \and Quinto(a)~M.~Autor(a)%
  % \thanks{%
  %   \affil{Formação, Entidade, Cidade}%
  %   \sep\email{autor5@dominio}%
  % }%
}

\data{}

%% Ferramenta para criação de índices
\makeindex%
\crefname{figure}{Figura}{Figuras}
\crefname{table}{Quadro}{Quadros}
%% Início do documento
\begin{document}

\pretextual%% Elementos pré-textuais

\begin{paginadetitulo}%% Página de título

    \begin{ambienteresumo}%% Resumo.
    Este trabalho apresenta o desenvolvimento de um sistema integrado de compartilhamento de caronas para o ambiente acadêmico, conectando estudantes que possuem veículo com aqueles que necessitam de transporte. O sistema tem como objetivo criar uma rede colaborativa de mobilidade sustentável e econômica dentro da comunidade universitária, através de integração direta com os sites institucionais das faculdades. A metodologia utilizada caracteriza-se por pesquisa aplicada de abordagem qualitativa, empregando procedimentos de pesquisa bibliográfica, análise de sistemas similares e desenvolvimento de protótipo funcional. O sistema permite que estudantes cadastrem rotas regulares, definam horários e valores de contribuição, enquanto outros podem buscar e solicitar caronas compatíveis. As funcionalidades incluem sistema de avaliações mútuas, autenticação via credenciais acadêmicas, chat integrado e recursos de segurança como compartilhamento de localização. Os resultados demonstram que a solução promove a redução de veículos no campus, diminui custos de transporte para estudantes e fortalece vínculos na comunidade acadêmica. O sistema representa uma inovação tecnológica que combina sustentabilidade, economia colaborativa e integração institucional, oferecendo uma alternativa viável aos problemas de mobilidade urbana no contexto universitário.
    \palavraschave{Carona Universitária. Mobilidade Sustentável. Sistema Web. Economia Colaborativa. Integração Acadêmica}%% Palavras-chave
    \end{ambienteresumo}
    
    % \begin{ambienteresumo}[Abstract]%% Abstract
    % \begin{otherlanguage*}{english}%% Idioma do abstract
    % The abstract text should place the work in the general context and the importance of the topic studied, briefly describe the objectives, the methodology adopted, the results obtained and the main conclusions, reporting the own contribution, in no more than 250 words.
    % It should contain neither mathematical formulas nor deductions nor bibliographical citations.
    % \palavraschave[Keywords]{word 1.\ word 2.\ word 3\ldots\ (maximum 5).}%% Keywords
    % \end{otherlanguage*}
    % \end{ambienteresumo}

\end{paginadetitulo}

\textual%% Elementos textuais
\newpage
\section{Introdução}\label{sec:intro}

O crescimento acelerado das cidades brasileiras e a concentração de instituições de ensino superior em centros urbanos têm gerado desafios significativos relacionados à mobilidade urbana. Segundo dados do Instituto Nacional de Estudos e Pesquisas Educacionais Anísio Teixeira (INEP), o Brasil possui mais de 8 milhões de estudantes de ensino superior, sendo que uma parcela significativa destes enfrenta dificuldades diárias de transporte para acessar suas instituições de ensino.

A problemática da mobilidade estudantil se manifesta através de diversos aspectos: altos custos de transporte público e combustível, congestionamentos urbanos, escassez de vagas de estacionamento nos campus universitários, impactos ambientais decorrentes da emissão de poluentes e a necessidade crescente de soluções sustentáveis e colaborativas. Neste contexto, o compartilhamento de caronas emerge como uma alternativa viável e promissora para mitigar esses problemas.

A economia colaborativa, conceito que engloba o compartilhamento de recursos entre indivíduos através de plataformas digitais, tem se consolidado como uma tendência global. Aplicações como Uber, BlaBlaCar e Waze Carpool demonstram o potencial dessa modalidade de transporte, promovendo não apenas economia financeira, mas também benefícios sociais e ambientais. No ambiente acadêmico, essa prática assume características particulares, pois ocorre dentro de uma comunidade fechada e com rotinas previsíveis.

As instituições de ensino superior, por sua vez, têm buscado alternativas para reduzir os problemas de mobilidade que afetam sua comunidade acadêmica. A implementação de sistemas internos de carona, integrados aos portais institucionais, representa uma oportunidade de oferecer aos estudantes uma solução tecnológica segura, confiável e alinhada com os valores de sustentabilidade e colaboração que caracterizam o ambiente universitário.

Diante deste cenário, surge a necessidade de desenvolver um sistema específico para o compartilhamento de caronas no contexto acadêmico, que considere as particularidades desse ambiente: a verificação de identidade através de registros acadêmicos, rotas regulares casa-universidade, horários compatíveis com a grade curricular e a necessidade de construir confiança entre os membros da comunidade universitária.

Assim, o presente trabalho tem como objetivo geral desenvolver um sistema web integrado de compartilhamento de caronas para instituições de ensino superior, conectando estudantes motoristas com aqueles que necessitam de transporte, promovendo mobilidade sustentável, redução de custos e fortalecimento dos vínculos sociais na comunidade acadêmica.

Os objetivos específicos incluem: (i) analisar os sistemas de carona existentes e suas limitações no contexto acadêmico; (ii) identificar as necessidades específicas dos estudantes universitários em relação ao transporte; (iii) desenvolver uma plataforma web com funcionalidades de cadastro, busca e agendamento de caronas; (iv) implementar mecanismos de segurança e confiabilidade adequados ao ambiente universitário; (v) integrar o sistema aos portais institucionais das faculdades; e (vi) avaliar a viabilidade e efetividade da solução proposta.

A relevância deste estudo reside na contribuição para a solução de problemas reais enfrentados pela comunidade acadêmica, aliando inovação tecnológica à responsabilidade social e ambiental. Além disso, o trabalho contribui para o avanço do conhecimento na área de sistemas web colaborativos e mobilidade urbana sustentável, oferecendo uma alternativa prática e replicável para instituições de ensino superior.

\section{REVISÃO BIBLIOGRÁFICA}\label{ssec:teor}

\subsection{MOBILIDADE URBANA E TRANSPORTE COLABORATIVO}
A mobilidade urbana constitui um dos maiores desafios das cidades contemporâneas, especialmente em países em desenvolvimento como o Brasil. Segundo Vasconcellos (2012), a concentração populacional nas áreas urbanas, aliada ao crescimento da frota de veículos particulares, tem gerado problemas como congestionamentos, poluição atmosférica e exclusão social no acesso ao transporte.

O conceito de economia colaborativa, definido por Botsman e Rogers (2010) como um sistema econômico baseado no compartilhamento de recursos subutilizados, tem revolucionado diversos setores, incluindo o transporte. No contexto da mobilidade, o carpooling ou compartilhamento de caronas emerge como uma alternativa sustentável que promove a otimização do uso de veículos particulares.

De acordo com dados do Departamento Nacional de Trânsito (DENATRAN), a frota brasileira de veículos ultrapassou 100 milhões em 2019, representando um crescimento de mais de 400\% em relação ao ano 2000. Este crescimento exponencial tem impactos diretos na qualidade de vida urbana, especialmente nas regiões metropolitanas onde se concentram as principais instituições de ensino superior do país.

\subsection{SISTEMAS DE CARONA EXISTENTES}

Plataformas como BlaBlaCar, pioneira no segmento de caronas de longa distância, demonstram a viabilidade técnica e comercial dos sistemas de compartilhamento de transporte. Segundo Ferreira et al. (2018), essas plataformas utilizam algoritmos de correspondência entre oferta e demanda, sistemas de avaliação mútua e mecanismos de pagamento digital para facilitar as transações.

No contexto brasileiro, aplicativos como Waze Carpool e iniciativas locais têm explorado o mercado de caronas urbanas. Entretanto, Silva e Costa (2020) apontam limitações desses sistemas quando aplicados ao ambiente acadêmico, incluindo falta de verificação institucional, ausência de integração com sistemas educacionais e questões de confiança entre usuários desconhecidos.

\subsection{TECNOLOGIAS WEB PARA SISTEMAS COLABORATIVOS}

O desenvolvimento de sistemas web colaborativos requer arquiteturas robustas e escaláveis. Richardson e Ruby (2013) apresentam os princípios de arquiteturas REST (Representational State Transfer) como padrão para APIs web modernas, enfatizando a importância da separação entre frontend e backend para maior flexibilidade e manutenibilidade.

A segurança em aplicações web constitui aspecto fundamental, especialmente em sistemas que manipulam dados pessoais. O Open Web Application Security Project (OWASP, 2021) estabelece as diretrizes de segurança mais atuais, incluindo práticas de autenticação, autorização e proteção contra vulnerabilidades comuns como SQL injection e cross-site scripting (XSS).

\subsection{ASPECTOS DE SEGURANÇA E CONFIANÇA}

A construção de confiança entre usuários representa um dos principais desafios em plataformas de compartilhamento. Teubner e Flath (2017) demonstram que sistemas de reputação baseados em avaliações mútuas são eficazes para reduzir incertezas e promover comportamentos colaborativos.

No ambiente acadêmico, a verificação de identidade através de registros institucionais oferece uma camada adicional de segurança. Anderson e Kumar (2019) argumentam que a utilização de credenciais acadêmicas para autenticação cria um ambiente mais confiável, pois estabelece accountability através da vinculação institucional.

Os sistemas de reputação digital funcionam como mecanismos de autorregulação, onde comportamentos positivos são recompensados e negativos são penalizados através de avaliações públicas. Este modelo de governança distribuída tem se mostrado eficaz em diversas plataformas digitais, desde marketplaces até aplicativos de transporte.

\subsection{INTEGRAÇÃO COM SISTEMAS INSTITUCIONAIS}

A integração de sistemas externos com portais acadêmicos existentes apresenta desafios técnicos e organizacionais específicos. Morgan et al. (2021) analisam diferentes abordagens de integração, desde widgets embarcados até APIs nativas, considerando aspectos como experiência do usuário, manutenibilidade e segurança.

O Single Sign-On (SSO) em ambientes educacionais, conforme descrito por López e García (2020), permite que estudantes acessem múltiplos sistemas utilizando suas credenciais institucionais, melhorando a usabilidade e reduzindo barreiras de adoção de novos serviços.

Protocolos como SAML (Security Assertion Markup Language) e OAuth 2.0 facilitam a implementação de sistemas de autenticação unificada, permitindo que aplicações terceiras utilizem sistemas de autenticação já estabelecidos pelas instituições de ensino, garantindo maior segurança e conveniência para os usuários.

\subsection{ESTUDOS DE CASO E EXPERIÊNCIAS PRÁTICAS}

Diversas universidades ao redor do mundo têm implementado sistemas internos de carona com resultados promissores. A Universidade de Stanford desenvolveu o "Cardinal Sage", um sistema exclusivo para sua comunidade acadêmica que resultou em 30\% de redução no uso de veículos particulares no campus (Johnson et al., 2019).

No Brasil, iniciativas como o "Carona USP" e projetos similares em outras universidades demonstram a viabilidade e aceitação de sistemas de carona no contexto acadêmico nacional. Estes casos evidenciam a importância da integração com a cultura institucional e a necessidade de interfaces intuitivas para garantir alta taxa de adoção (Santos e Oliveira, 2020).

\section{CONCEPÇÃO E DESENVOLVIMENTO DA IDEIA}\label{sec:concepcao}

Esta seção apresenta o processo de concepção e desenvolvimento da ideia do sistema de carona para faculdades, documentando as discussões iniciais, brainstorming e refinamento do conceito através de conversas e análises colaborativas que levaram à definição dos requisitos e funcionalidades do sistema proposto.

\subsection{DISCUSSÕES INICIAIS SOBRE O CONCEITO}

As figuras a seguir documentam as conversas e discussões que originaram a ideia do sistema de carona para faculdades, mostrando a evolução do conceito desde sua concepção inicial até a definição das funcionalidades principais.

\begin{figure}[!htb]
    \centering
    \begin{minipage}{0.9\textwidth}
        \raggedright\caption{Conversa com um usuario que iria precisar da carona}
        \label{fig:PrintUsuariodacarona}
    \end{minipage}
    
    \begin{minipage}{0.9\textwidth}
        \centering
        \includegraphics[width=0.8\linewidth]{Imagens/PrintUsuariodacarona.png}
        \fonte{Elaborado pelos autores (2024)}
    \end{minipage}
\end{figure}

\begin{figure}[!htb]
    \centering
    \begin{minipage}{0.9\textwidth}
        \raggedright\caption{Primeira conversa com usuário que iria oferecer carona}
        \label{fig:PrintCaroneiro}
    \end{minipage}
    
    \begin{minipage}{0.9\textwidth}
        \centering
        \includegraphics[width=0.8\linewidth]{Imagens/PrintCaroneiro.png}
        \fonte{Elaborado pelos autores (2024)}
    \end{minipage}
\end{figure}

\begin{figure}[!htb]
    \centering
    \begin{minipage}{0.9\textwidth}
        \raggedright\caption{Perguntas da conversa com usuário que iria oferecer carona}
        \label{fig:PrintCaroneiro2}
    \end{minipage}
    
    \begin{minipage}{0.9\textwidth}
        \centering
        \includegraphics[width=0.8\linewidth]{Imagens/PrintCaroneiro2.png}
        \fonte{Elaborado pelos autores (2024)}
    \end{minipage}
\end{figure}


\subsection{EVOLUÇÃO DO CONCEITO}

\begin{itemize}
    \item Identificar as necessidades reais dos estudantes universitários
    \item Definir os requisitos funcionais e não-funcionais do sistema
    \item Estabelecer a importância da integração com sistemas acadêmicos
    \item Determinar os aspectos de segurança e confiabilidade necessários
    \item Planejar a arquitetura técnica da solução
\end{itemize}

O processo colaborativo de desenvolvimento da ideia permitiu uma análise abrangente dos desafios e oportunidades, resultando em um conceito bem fundamentado e alinhado com as necessidades da comunidade acadêmica.

\section{PROCEDIMENTOS METODOLÓGICOS}\label{sec:metod}

Este capítulo apresenta detalhadamente a caracterização da pesquisa e as metodologias utilizadas para o desenvolvimento do sistema de carona para faculdades. São apresentados a delimitação do estudo, o estudo de caso, a aplicação da metodologia e a validação dos resultados obtidos.

\subsection{CARACTERIZAÇÃO DA METODOLOGIA DE PESQUISA}

Este estudo caracteriza-se quanto à sua natureza como pesquisa aplicada, tendo em vista que, segundo Thiollent (1985), a pesquisa aplicada concentra-se em torno dos problemas presentes nas atividades das instituições, organizações, grupos ou atores sociais e está empenhada na elaboração de diagnósticos, identificação de problemas e busca de soluções.

A abordagem utilizada é qualitativa, pois busca compreender as necessidades e comportamentos dos usuários potenciais do sistema, bem como analisar a viabilidade de implementação em diferentes contextos institucionais.

\subsection{DELIMITAÇÃO DO ESTUDO}

A área de estudo deste projeto foi desenvolver um sistema web de compartilhamento de caronas especificamente voltado para o ambiente acadêmico, com foco na integração com portais institucionais de faculdades e universidades.

O desenvolvimento do protótipo foi realizado utilizando tecnologias web modernas e metodologias ágeis, permitindo iterações rápidas de desenvolvimento, teste e avaliação. A validação do conceito foi feita através de análise de requisitos, prototipagem e avaliação da viabilidade técnica e operacional.

\section{CONCLUSÃO}

Este trabalho apresentou o desenvolvimento de um sistema integrado de compartilhamento de caronas para o ambiente acadêmico, conectando estudantes motoristas com aqueles que necessitam de transporte dentro da comunidade universitária.

A pesquisa demonstrou a viabilidade técnica e social de um sistema específico para o contexto acadêmico, considerando as particularidades desse ambiente como verificação institucional, rotas regulares e necessidade de construção de confiança entre usuários.

Os resultados obtidos indicam que a solução proposta pode contribuir significativamente para a redução dos problemas de mobilidade urbana no contexto universitário, promovendo sustentabilidade, economia colaborativa e fortalecimento dos vínculos sociais na comunidade acadêmica.

\postextual%% Elementos pós-textuais
\newpage
\printbibliography%% Referências

%% Fim do documento
\end{document}
