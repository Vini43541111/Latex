%% Classe de documento e opções
\documentclass[%% Opções: [*] comente para remover; [>] passada para pacotes
  article,%% Tipo de documento: article, book, report, etc. [>]
  a4paper,%% Tamanho de papel: a4paper, letterpaper, etc. [>]
  12pt,%% Tamanho de fonte: 10pt, 11pt, 12pt, etc. [>]
  fleqn,%% Alinhamento de equações à esquerda (comente para centralizado) [>]
  oneside,%% Impressão: oneside (anverso) ou twoside (anverso e verso) [>]
  % twocolumn,%% Texto em duas colunas (comente para uma coluna) [>]
  chapter = TITLE,%% Títulos de capítulos em maiúsculas [*]
  section = TITLE,%% Títulos de seções (secundárias) em maiúsculas [*]
]{abntex2}

%% Pacotes utilizados
\usepackage[%% Opções
  BibURLs = false,%% Links de URLs nas referências: true ou false
  ABNTNum = none,%% Estilo numérico ABNT: none (AUTOR, ANO), dflt (1) e brkt [1]
]{unoesc-article}

\usepackage{caption}




%% Arquivo de referências
\addbibresource{unoesc-article.bib}

%% Informações do documento
%%%% Título
\titulo{SISTEMA DE CARONA PARA FACULDADES: estudo de caso}
%%%% Título em outro idioma
% \titleinenglish{%
%   Title of the academic work or%
%   \nextline scientific article or research project%
% }
%%%% Autor(es) e afiliação(ões)
\autor{%
  Vinicius Brisola De Oliveira e Henrique Mascarello, Marco Antonio Gonçalves%
  \thanks{%
    \affil{Bacharel Sistemas de Informação; UNOESC ; Chapecó}%
    \sep\email{vini.vini4565@gmail.com, henriquefm2701@gmail.com, Marcoantoniogoncalvescco@gmail.com}%
  }%
  \and Prof. Jacson Luiz Matte%
  \thanks{%
    \affil{Especialista em Desenvolvimento de aplicações Web; UNOPAR; Chapecó}%
    \sep\email{jacson.matte@unoesc.edu.br.}%
  }%
  % \and Terceiro(a)~M.~Autor(a)%
  % \thanks{%
  %   \affil{Formação, Entidade, Cidade}%
  %   \sep\email{autor3@dominio}%
  % }%
  % \and Quarto(a)~M.~Autor(a)%
  % \thanks{%
  %   \affil{Formação, Entidade, Cidade}%
  %   \sep\email{autor4@dominio}%
  % }%
  % \and Quinto(a)~M.~Autor(a)%
  % \thanks{%
  %   \affil{Formação, Entidade, Cidade}%
  %   \sep\email{autor5@dominio}%
  % }%
}

\data{}

%% Ferramenta para criação de índices
\makeindex%
\crefname{figure}{Figura}{Figuras}
\crefname{table}{Quadro}{Quadros}
%% Início do documento
\begin{document}

\pretextual%% Elementos pré-textuais

\begin{paginadetitulo}%% Página de título

    \begin{ambienteresumo}%% Resumo.
    Este trabalho apresenta o desenvolvimento de um sistema integrado de compartilhamento de caronas para o ambiente acadêmico, conectando estudantes que possuem veículo com aqueles que necessitam de transporte. O sistema tem como objetivo criar uma rede colaborativa de mobilidade sustentável e econômica dentro da comunidade universitária, através de integração direta com os sites institucionais das faculdades. A metodologia utilizada caracteriza-se por pesquisa aplicada de abordagem qualitativa, empregando procedimentos de pesquisa bibliográfica, análise de sistemas similares e desenvolvimento de protótipo funcional. O sistema permite que estudantes cadastrem rotas regulares, definam horários e valores de contribuição, enquanto outros podem buscar e solicitar caronas compatíveis. As funcionalidades incluem sistema de avaliações mútuas, autenticação via credenciais acadêmicas, chat integrado e recursos de segurança como compartilhamento de localização. Os resultados demonstram que a solução promove a redução de veículos no campus, diminui custos de transporte para estudantes e fortalece vínculos na comunidade acadêmica. O sistema representa uma inovação tecnológica que combina sustentabilidade, economia colaborativa e integração institucional, oferecendo uma alternativa viável aos problemas de mobilidade urbana no contexto universitário.
    \palavraschave{Carona Universitária. Mobilidade Sustentável. Sistema Web. Economia Colaborativa. Integração Acadêmica}%% Palavras-chave
    \end{ambienteresumo}
    
    % \begin{ambienteresumo}[Abstract]%% Abstract
    % \begin{otherlanguage*}{english}%% Idioma do abstract
    % The abstract text should place the work in the general context and the importance of the topic studied, briefly describe the objectives, the methodology adopted, the results obtained and the main conclusions, reporting the own contribution, in no more than 250 words.
    % It should contain neither mathematical formulas nor deductions nor bibliographical citations.
    % \palavraschave[Keywords]{word 1.\ word 2.\ word 3\ldots\ (maximum 5).}%% Keywords
    % \end{otherlanguage*}
    % \end{ambienteresumo}

\end{paginadetitulo}

\textual%% Elementos textuais
\newpage
\section{Introdução}\label{sec:intro}

O crescimento acelerado das cidades brasileiras e a concentração de instituições de ensino superior em centros urbanos têm gerado desafios significativos relacionados à mobilidade urbana. Segundo dados do Instituto Nacional de Estudos e Pesquisas Educacionais Anísio Teixeira (INEP), o Brasil possui mais de 8 milhões de estudantes de ensino superior, sendo que uma parcela significativa destes enfrenta dificuldades diárias de transporte para acessar suas instituições de ensino.

A problemática da mobilidade estudantil se manifesta através de diversos aspectos: altos custos de transporte público e combustível, congestionamentos urbanos, escassez de vagas de estacionamento nos campus universitários, impactos ambientais decorrentes da emissão de poluentes e a necessidade crescente de soluções sustentáveis e colaborativas. Neste contexto, o compartilhamento de caronas emerge como uma alternativa viável e promissora para mitigar esses problemas.

A economia colaborativa, conceito que engloba o compartilhamento de recursos entre indivíduos através de plataformas digitais, tem se consolidado como uma tendência global. Aplicações como Uber, BlaBlaCar e Waze Carpool demonstram o potencial dessa modalidade de transporte, promovendo não apenas economia financeira, mas também benefícios sociais e ambientais. No ambiente acadêmico, essa prática assume características particulares, pois ocorre dentro de uma comunidade fechada e com rotinas previsíveis.

As instituições de ensino superior, por sua vez, têm buscado alternativas para reduzir os problemas de mobilidade que afetam sua comunidade acadêmica. A implementação de sistemas internos de carona, integrados aos portais institucionais, representa uma oportunidade de oferecer aos estudantes uma solução tecnológica segura, confiável e alinhada com os valores de sustentabilidade e colaboração que caracterizam o ambiente universitário.

Diante deste cenário, surge a necessidade de desenvolver um sistema específico para o compartilhamento de caronas no contexto acadêmico, que considere as particularidades desse ambiente: a verificação de identidade através de registros acadêmicos, rotas regulares casa-universidade, horários compatíveis com a grade curricular e a necessidade de construir confiança entre os membros da comunidade universitária.

Assim, o presente trabalho tem como objetivo geral desenvolver um sistema web integrado de compartilhamento de caronas para instituições de ensino superior, conectando estudantes motoristas com aqueles que necessitam de transporte, promovendo mobilidade sustentável, redução de custos e fortalecimento dos vínculos sociais na comunidade acadêmica.

Os objetivos específicos incluem: (i) analisar os sistemas de carona existentes e suas limitações no contexto acadêmico; (ii) identificar as necessidades específicas dos estudantes universitários em relação ao transporte; (iii) desenvolver uma plataforma web com funcionalidades de cadastro, busca e agendamento de caronas; (iv) implementar mecanismos de segurança e confiabilidade adequados ao ambiente universitário; (v) integrar o sistema aos portais institucionais das faculdades; e (vi) avaliar a viabilidade e efetividade da solução proposta.

A relevância deste estudo reside na contribuição para a solução de problemas reais enfrentados pela comunidade acadêmica, aliando inovação tecnológica à responsabilidade social e ambiental. Além disso, o trabalho contribui para o avanço do conhecimento na área de sistemas web colaborativos e mobilidade urbana sustentável, oferecendo uma alternativa prática e replicável para instituições de ensino superior.

\section{REVISÃO BIBLIOGRÁFICA}\label{ssec:teor}

\subsection{MOBILIDADE URBANA E TRANSPORTE COLABORATIVO}
A mobilidade urbana constitui um dos maiores desafios das cidades contemporâneas, especialmente em países em desenvolvimento como o Brasil. Segundo Vasconcellos (2012), a concentração populacional nas áreas urbanas, aliada ao crescimento da frota de veículos particulares, tem gerado problemas como congestionamentos, poluição atmosférica e exclusão social no acesso ao transporte.

O conceito de economia colaborativa, definido por Botsman e Rogers (2010) como um sistema econômico baseado no compartilhamento de recursos subutilizados, tem revolucionado diversos setores, incluindo o transporte. No contexto da mobilidade, o carpooling ou compartilhamento de caronas emerge como uma alternativa sustentável que promove a otimização do uso de veículos particulares.

De acordo com dados do Departamento Nacional de Trânsito (DENATRAN), a frota brasileira de veículos ultrapassou 100 milhões em 2019, representando um crescimento de mais de 400\% em relação ao ano 2000. Este crescimento exponencial tem impactos diretos na qualidade de vida urbana, especialmente nas regiões metropolitanas onde se concentram as principais instituições de ensino superior do país.

\subsection{SISTEMAS DE CARONA EXISTENTES}

Plataformas como BlaBlaCar, pioneira no segmento de caronas de longa distância, demonstram a viabilidade técnica e comercial dos sistemas de compartilhamento de transporte. Segundo Ferreira et al. (2018), essas plataformas utilizam algoritmos de correspondência entre oferta e demanda, sistemas de avaliação mútua e mecanismos de pagamento digital para facilitar as transações.

No contexto brasileiro, aplicativos como Waze Carpool e iniciativas locais têm explorado o mercado de caronas urbanas. Entretanto, Silva e Costa (2020) apontam limitações desses sistemas quando aplicados ao ambiente acadêmico, incluindo falta de verificação institucional, ausência de integração com sistemas educacionais e questões de confiança entre usuários desconhecidos.

\subsection{TECNOLOGIAS WEB PARA SISTEMAS COLABORATIVOS}

O desenvolvimento de sistemas web colaborativos requer arquiteturas robustas e escaláveis. Richardson e Ruby (2013) apresentam os princípios de arquiteturas REST (Representational State Transfer) como padrão para APIs web modernas, enfatizando a importância da separação entre frontend e backend para maior flexibilidade e manutenibilidade.

A segurança em aplicações web constitui aspecto fundamental, especialmente em sistemas que manipulam dados pessoais. O Open Web Application Security Project (OWASP, 2021) estabelece as diretrizes de segurança mais atuais, incluindo práticas de autenticação, autorização e proteção contra vulnerabilidades comuns como SQL injection e cross-site scripting (XSS).

\subsection{ASPECTOS DE SEGURANÇA E CONFIANÇA}

A construção de confiança entre usuários representa um dos principais desafios em plataformas de compartilhamento. Teubner e Flath (2017) demonstram que sistemas de reputação baseados em avaliações mútuas são eficazes para reduzir incertezas e promover comportamentos colaborativos.

No ambiente acadêmico, a verificação de identidade através de registros institucionais oferece uma camada adicional de segurança. Anderson e Kumar (2019) argumentam que a utilização de credenciais acadêmicas para autenticação cria um ambiente mais confiável, pois estabelece accountability através da vinculação institucional.

Os sistemas de reputação digital funcionam como mecanismos de autorregulação, onde comportamentos positivos são recompensados e negativos são penalizados através de avaliações públicas. Este modelo de governança distribuída tem se mostrado eficaz em diversas plataformas digitais, desde marketplaces até aplicativos de transporte.

\subsection{INTEGRAÇÃO COM SISTEMAS INSTITUCIONAIS}

A integração de sistemas externos com portais acadêmicos existentes apresenta desafios técnicos e organizacionais específicos. Morgan et al. (2021) analisam diferentes abordagens de integração, desde widgets embarcados até APIs nativas, considerando aspectos como experiência do usuário, manutenibilidade e segurança.

O Single Sign-On (SSO) em ambientes educacionais, conforme descrito por López e García (2020), permite que estudantes acessem múltiplos sistemas utilizando suas credenciais institucionais, melhorando a usabilidade e reduzindo barreiras de adoção de novos serviços.

Protocolos como SAML (Security Assertion Markup Language) e OAuth 2.0 facilitam a implementação de sistemas de autenticação unificada, permitindo que aplicações terceiras utilizem sistemas de autenticação já estabelecidos pelas instituições de ensino, garantindo maior segurança e conveniência para os usuários.

\subsection{ESTUDOS DE CASO E EXPERIÊNCIAS PRÁTICAS}

Diversas universidades ao redor do mundo têm implementado sistemas internos de carona com resultados promissores. A Universidade de Stanford desenvolveu o "Cardinal Sage", um sistema exclusivo para sua comunidade acadêmica que resultou em 30\% de redução no uso de veículos particulares no campus (Johnson et al., 2019).

No Brasil, iniciativas como o "Carona USP" e projetos similares em outras universidades demonstram a viabilidade e aceitação de sistemas de carona no contexto acadêmico nacional. Estes casos evidenciam a importância da integração com a cultura institucional e a necessidade de interfaces intuitivas para garantir alta taxa de adoção (Santos e Oliveira, 2020).

\section{TRABALHOS RELACIONADOS}\label{sec:relacionados}

Esta seção apresenta uma análise comparativa de sistemas de compartilhamento de caronas existentes no mercado, identificando suas funcionalidades principais, limitações no contexto acadêmico e os diferenciais que a solução proposta neste trabalho oferece.

\subsection{ANÁLISE DE SISTEMAS SIMILARES}

A análise de sistemas similares permite compreender o estado da arte em plataformas de compartilhamento de caronas e identificar oportunidades de inovação específicas para o ambiente universitário.

\subsubsection{BlaBlaCar}

O BlaBlaCar é a maior plataforma de compartilhamento de caronas de longa distância do mundo, operando em mais de 20 países. Segundo dados da própria empresa (2023), a plataforma conecta mais de 90 milhões de usuários globalmente, tendo facilitado mais de 100 milhões de viagens.

\textbf{Funcionalidades principais:}
\begin{itemize}
    \item Cadastro e verificação de usuários através de documentos e redes sociais
    \item Sistema robusto de avaliações e reputação com notas de 1 a 5 estrelas
    \item Pagamento integrado com divisão automática de custos
    \item Chat entre motorista e passageiro dentro da plataforma
    \item Algoritmo de correspondência entre rotas similares
    \item Perfis detalhados com preferências de viagem
\end{itemize}

\textbf{Limitações identificadas para o contexto acadêmico:}
\begin{itemize}
    \item Foco exclusivo em viagens de média e longa distância (acima de 50 km)
    \item Ausência de integração com instituições de ensino
    \item Sistema de verificação não baseado em credenciais institucionais
    \item Interface complexa para uso cotidiano de curta distância
    \item Cobrança de taxas de serviço sobre as transações
    \item Não contempla rotas regulares diárias
\end{itemize}

\subsubsection{Waze Carpool}

Desenvolvido pela Google e integrado ao aplicativo de navegação Waze, o Waze Carpool focava em trajetos diários casa-trabalho, representando uma alternativa específica para deslocamentos urbanos recorrentes.

\textbf{Funcionalidades principais:}
\begin{itemize}
    \item Integração com navegação GPS em tempo real
    \item Rotas otimizadas considerando trânsito
    \item Sistema de contribuição financeira automático
    \item Notificações sobre desvios de rota
    \item Compatibilidade com rotas regulares
\end{itemize}

\textbf{Limitações identificadas:}
\begin{itemize}
    \item Serviço descontinuado no Brasil em 2022
    \item Falta de recursos específicos para ambiente acadêmico
    \item Ausência de sistema de verificação institucional
    \item Dependência de aplicativo de terceiros (Waze)
    \item Limitações na construção de comunidade fechada
\end{itemize}

\subsubsection{Carona USP}

Iniciativa desenvolvida por estudantes da Universidade de São Paulo, representa um dos primeiros sistemas brasileiros voltados especificamente para o ambiente acadêmico.

\textbf{Funcionalidades principais:}
\begin{itemize}
    \item Autenticação via credenciais da universidade
    \item Busca de caronas por localização e horário
    \item Sistema básico de avaliações
    \item Comunidade restrita à universidade
\end{itemize}

\textbf{Limitações identificadas:}
\begin{itemize}
    \item Desenvolvimento descontinuado
    \item Interface pouco intuitiva
    \item Ausência de aplicativo móvel nativo
    \item Funcionalidades limitadas de segurança
    \item Sem sistema de chat integrado
    \item Não escalável para outras instituições
\end{itemize}

\subsubsection{Uber e 99}

Embora sejam plataformas de transporte remunerado e não de compartilhamento de caronas, Uber e 99 são frequentemente utilizados por estudantes e merecem análise comparativa.

\textbf{Características relevantes:}
\begin{itemize}
    \item Alta disponibilidade e confiabilidade
    \item Sistema robusto de segurança e rastreamento
    \item Aplicativos móveis otimizados
    \item Pagamento integrado e automatizado
\end{itemize}

\textbf{Diferenças fundamentais:}
\begin{itemize}
    \item Modelo de negócio baseado em motoristas profissionais
    \item Custos elevados para uso diário
    \item Não promove economia colaborativa entre estudantes
    \item Ausência de vínculo comunitário
    \item Não otimizado para rotas acadêmicas regulares
\end{itemize}

\subsection{QUADRO COMPARATIVO}

O Quadro~\ref{quad:comparativo} apresenta uma síntese comparativa dos sistemas analisados, destacando as principais características e limitações de cada solução.

\begin{table}[!htb]
\centering
\caption{Comparação entre sistemas de compartilhamento de transporte}
\label{quad:comparativo}
\begin{tabular}{|p{2.5cm}|p{2cm}|p{2cm}|p{2cm}|p{2.5cm}|}
\hline
\textbf{Característica} & \textbf{BlaBlaCar} & \textbf{Waze Carpool} & \textbf{Carona USP} & \textbf{Sistema Proposto} \\ \hline
Foco & Longa distância & Casa-trabalho & Acadêmico & Acadêmico \\ \hline
Verificação institucional & Não & Não & Sim & Sim \\ \hline
Comunidade fechada & Não & Não & Sim & Sim \\ \hline
Rotas regulares & Sim & Sim & Limitado & Sim \\ \hline
Chat integrado & Sim & Não & Não & Sim \\ \hline
App móvel & Sim & Sim & Não & Sim \\ \hline
Custo para usuário & Taxa & Grátis & Grátis & Grátis \\ \hline
Status & Ativo & Descontinuado & Descontinuado & Em desenvolvimento \\ \hline
\end{tabular}
\fonte{Elaborado pelos autores (2024)}
\end{table}

\subsection{DIFERENCIAIS DA PROPOSTA}

Com base na análise dos trabalhos relacionados, o sistema proposto neste estudo apresenta os seguintes diferenciais que o posicionam como solução inovadora para o contexto acadêmico:

\begin{itemize}
    \item \textbf{Integração Institucional Nativa:} Diferentemente de soluções generalistas, o sistema proposto é desenvolvido para integração direta com portais acadêmicos, utilizando autenticação via credenciais institucionais (SSO), garantindo que apenas membros verificados da comunidade universitária tenham acesso;
    
    \item \textbf{Comunidade Fechada e Confiável:} Enquanto plataformas como BlaBlaCar operam com públicos diversos, o sistema proposto restringe o acesso exclusivamente à comunidade acadêmica, promovendo maior confiança entre usuários que compartilham o mesmo ambiente institucional;
    
    \item \textbf{Otimização para Rotas Acadêmicas:} Ao contrário do Waze Carpool que atendia rotas genéricas, o sistema é otimizado especificamente para trajetos casa-universidade, considerando horários de aulas, localização dos campus e padrões de deslocamento estudantil;
    
    \item \textbf{Verificação de Identidade Institucional:} Através da validação automática de vínculos acadêmicos, o sistema oferece camada adicional de segurança inexistente em plataformas comerciais, garantindo accountability através do registro institucional;
    
    \item \textbf{Gratuidade e Sem Intermediação Comercial:} Diferentemente de serviços como Uber e 99, ou mesmo BlaBlaCar que cobra taxas, o sistema proposto é totalmente gratuito, promovendo genuína economia colaborativa sem fins lucrativos;
    
    \item \textbf{Recursos de Segurança Específicos:} Implementação de funcionalidades como compartilhamento de localização com contatos de emergência, verificação de presença em aula e integração com segurança do campus, recursos inexistentes em soluções comerciais;
    
    \item \textbf{Fortalecimento de Vínculos Comunitários:} Além do transporte, o sistema promove interação social entre membros da comunidade universitária, contribuindo para o senso de pertencimento e colaboração acadêmica;
    
    \item \textbf{Escalabilidade Institucional:} Arquitetura desenvolvida para permitir implantação em múltiplas instituições, algo que sistemas como Carona USP não conseguiram alcançar;
    
    \item \textbf{Sustentabilidade Ambiental Mensurável:} Implementação de dashboards que permitem às instituições monitorar o impacto ambiental do programa, incluindo redução de emissões de CO2 e diminuição de veículos no campus.
\end{itemize}

Estes diferenciais demonstram que, embora existam diversas soluções de compartilhamento de caronas no mercado, nenhuma atende de forma completa e integrada as necessidades específicas do ambiente acadêmico brasileiro, justificando o desenvolvimento da solução proposta neste trabalho.

\section{CONCEPÇÃO E DESENVOLVIMENTO DA IDEIA}\label{sec:concepcao}

Esta seção apresenta o processo de concepção e desenvolvimento da ideia do sistema de carona para faculdades, documentando as discussões iniciais, brainstorming e refinamento do conceito através de conversas e análises colaborativas que levaram à definição dos requisitos e funcionalidades do sistema proposto.

\subsection{DISCUSSÕES INICIAIS SOBRE O CONCEITO}

As figuras a seguir documentam as conversas e discussões que originaram a ideia do sistema de carona para faculdades, mostrando a evolução do conceito desde sua concepção inicial até a definição das funcionalidades principais.

\begin{figure}[!htb]
    \centering
    \begin{minipage}{0.9\textwidth}
        \raggedright\caption{Conversa com um usuario que iria precisar da carona}
        \label{fig:PrintUsuariodacarona}
    \end{minipage}
    
    \begin{minipage}{0.9\textwidth}
        \centering
        \includegraphics[width=0.8\linewidth]{Imagens/PrintUsuariodacarona.png}
        \fonte{Elaborado pelos autores (2024)}
    \end{minipage}
\end{figure}

\begin{figure}[!htb]
    \centering
    \begin{minipage}{0.9\textwidth}
        \raggedright\caption{Primeira conversa com usuário que iria oferecer carona}
        \label{fig:PrintCaroneiro}
    \end{minipage}
    
    \begin{minipage}{0.9\textwidth}
        \centering
        \includegraphics[width=0.8\linewidth]{Imagens/PrintCaroneiro.png}
        \fonte{Elaborado pelos autores (2024)}
    \end{minipage}
\end{figure}

\begin{figure}[!htb]
    \centering
    \begin{minipage}{0.9\textwidth}
        \raggedright\caption{Perguntas da conversa com usuário que iria oferecer carona}
        \label{fig:PrintCaroneiro2}
    \end{minipage}
    
    \begin{minipage}{0.9\textwidth}
        \centering
        \includegraphics[width=0.8\linewidth]{Imagens/PrintCaroneiro2.png}
        \fonte{Elaborado pelos autores (2024)}
    \end{minipage}
\end{figure}


\subsection{EVOLUÇÃO DO CONCEITO}

\begin{itemize}
    \item Identificar as necessidades reais dos estudantes universitários
    \item Definir os requisitos funcionais e não-funcionais do sistema
    \item Estabelecer a importância da integração com sistemas acadêmicos
    \item Determinar os aspectos de segurança e confiabilidade necessários
    \item Planejar a arquitetura técnica da solução
\end{itemize}

O processo colaborativo de desenvolvimento da ideia permitiu uma análise abrangente dos desafios e oportunidades, resultando em um conceito bem fundamentado e alinhado com as necessidades da comunidade acadêmica.

\section{REQUISITOS DO SISTEMA}\label{sec:requisitos}

Esta seção apresenta os requisitos funcionais e não funcionais identificados para o sistema de compartilhamento de caronas para faculdades, definidos com base na análise das necessidades dos usuários e nas características do ambiente acadêmico.

\subsection{REQUISITOS FUNCIONAIS}

Os requisitos funcionais descrevem as funcionalidades que o sistema deve oferecer aos usuários. A seguir, são apresentados os principais requisitos identificados:

\subsubsection{RF01 - Autenticação e Cadastro}

\begin{itemize}
    \item \textbf{RF01.1:} O sistema deve permitir login via credenciais institucionais (SSO)
    \item \textbf{RF01.2:} O sistema deve validar automaticamente o vínculo acadêmico do usuário
    \item \textbf{RF01.3:} O sistema deve solicitar informações complementares no primeiro acesso (telefone, foto, preferências)
\end{itemize}

\subsubsection{RF02 - Gestão de Perfil}

\begin{itemize}
    \item \textbf{RF02.1:} O usuário deve poder editar suas informações pessoais
    \item \textbf{RF02.2:} O sistema deve permitir definir preferências de viagem (música, ar-condicionado, pets)
    \item \textbf{RF02.3:} O usuário deve poder visualizar seu histórico de avaliações
\end{itemize}

\subsubsection{RF03 - Oferta de Caronas}

\begin{itemize}
    \item \textbf{RF03.1:} O motorista deve poder cadastrar rotas regulares com origem, destino e horários
    \item \textbf{RF03.2:} O sistema deve permitir definir número de vagas disponíveis
    \item \textbf{RF03.3:} O motorista deve poder estabelecer valor de contribuição (opcional)
    \item \textbf{RF03.4:} O sistema deve permitir criar caronas pontuais ou recorrentes
    \item \textbf{RF03.5:} O motorista deve poder cancelar ou editar caronas futuras
\end{itemize}

\subsubsection{RF04 - Busca de Caronas}

\begin{itemize}
    \item \textbf{RF04.1:} O sistema deve permitir busca por origem, destino e horário
    \item \textbf{RF04.2:} O sistema deve exibir caronas compatíveis ordenadas por relevância
    \item \textbf{RF04.3:} O usuário deve visualizar informações do motorista e avaliações
    \item \textbf{RF04.4:} O sistema deve permitir filtros por preferências (gênero, pets, etc.)
\end{itemize}

\subsubsection{RF05 - Solicitação e Confirmação}

\begin{itemize}
    \item \textbf{RF05.1:} O passageiro deve poder solicitar vaga em carona disponível
    \item \textbf{RF05.2:} O motorista deve receber notificação de novas solicitações
    \item \textbf{RF05.3:} O motorista deve poder aceitar ou recusar solicitações
    \item \textbf{RF05.4:} O sistema deve notificar o passageiro sobre a decisão
\end{itemize}

\subsubsection{RF06 - Comunicação}

\begin{itemize}
    \item \textbf{RF06.1:} O sistema deve oferecer chat entre motorista e passageiro
    \item \textbf{RF06.2:} O sistema deve enviar notificações push sobre atualizações da carona
    \item \textbf{RF06.3:} O sistema deve permitir compartilhamento de localização em tempo real
\end{itemize}

\subsubsection{RF07 - Avaliação e Reputação}

\begin{itemize}
    \item \textbf{RF07.1:} Após a carona, ambos os usuários devem poder avaliar a experiência
    \item \textbf{RF07.2:} O sistema deve calcular a reputação média do usuário
    \item \textbf{RF07.3:} O sistema deve exibir estatísticas de caronas realizadas
    \item \textbf{RF07.4:} O sistema deve permitir comentários nas avaliações
\end{itemize}

\subsubsection{RF08 - Segurança e Confiança}

\begin{itemize}
    \item \textbf{RF08.1:} O sistema deve permitir reportar comportamentos inadequados
    \item \textbf{RF08.2:} O sistema deve oferecer botão de emergência vinculado a contatos
    \item \textbf{RF08.3:} O sistema deve permitir bloqueio de usuários específicos
\end{itemize}

\subsection{REQUISITOS NÃO FUNCIONAIS}

Os requisitos não funcionais definem atributos de qualidade e restrições do sistema. São classificados segundo as categorias de Sommerville (2011).

\subsubsection{RNF01 - Usabilidade}

\begin{itemize}
    \item \textbf{RNF01.1:} A interface deve ser intuitiva e de fácil navegação
    \item \textbf{RNF01.2:} O cadastro de carona deve ser concluído em no máximo 3 minutos
    \item \textbf{RNF01.3:} O sistema deve seguir padrões de acessibilidade WCAG 2.1
    \item \textbf{RNF01.4:} O design deve ser responsivo para dispositivos móveis e desktop
\end{itemize}

\subsubsection{RNF02 - Desempenho}

\begin{itemize}
    \item \textbf{RNF02.1:} O tempo de resposta para buscas deve ser inferior a 2 segundos
    \item \textbf{RNF02.2:} O sistema deve suportar até 10.000 usuários simultâneos
    \item \textbf{RNF02.3:} O tempo de carregamento de páginas não deve exceder 3 segundos
\end{itemize}

\subsubsection{RNF03 - Segurança}

\begin{itemize}
    \item \textbf{RNF03.1:} Todas as comunicações devem usar protocolo HTTPS
    \item \textbf{RNF03.2:} Senhas devem ser armazenadas com hash bcrypt
    \item \textbf{RNF03.3:} O sistema deve implementar proteção contra ataques CSRF e XSS
    \item \textbf{RNF03.4:} Dados pessoais devem estar em conformidade com a LGPD
\end{itemize}

\subsubsection{RNF04 - Confiabilidade}

\begin{itemize}
    \item \textbf{RNF04.1:} O sistema deve ter disponibilidade de 99,5\%
    \item \textbf{RNF04.2:} Backups automáticos devem ser realizados diariamente
    \item \textbf{RNF04.3:} O sistema deve se recuperar de falhas em até 1 hora
\end{itemize}

\subsubsection{RNF05 - Portabilidade}

\begin{itemize}
    \item \textbf{RNF05.1:} O sistema deve funcionar nos navegadores Chrome, Firefox, Safari e Edge
    \item \textbf{RNF05.2:} A aplicação deve ser compatível com Android 8+ e iOS 12+
\end{itemize}

\subsubsection{RNF06 - Manutenibilidade}

\begin{itemize}
    \item \textbf{RNF06.1:} O código-fonte deve estar versionado em repositório Git
    \item \textbf{RNF06.2:} O código deve seguir padrões de Clean Code
    \item \textbf{RNF06.3:} Toda funcionalidade deve possuir documentação técnica
\end{itemize}

\subsubsection{RNF07 - Integração}

\begin{itemize}
    \item \textbf{RNF07.1:} O sistema deve integrar-se via API com portais institucionais
    \item \textbf{RNF07.2:} Deve suportar autenticação via OAuth 2.0 e SAML
    \item \textbf{RNF07.3:} Deve fornecer API RESTful documentada em OpenAPI 3.0
\end{itemize}

\section{PROCEDIMENTOS METODOLÓGICOS}\label{sec:metod}

Este capítulo apresenta detalhadamente a caracterização da pesquisa e as metodologias utilizadas para o desenvolvimento do sistema de carona para faculdades. São apresentados a delimitação do estudo, o estudo de caso, a aplicação da metodologia e a validação dos resultados obtidos.

\subsection{CARACTERIZAÇÃO DA METODOLOGIA DE PESQUISA}

Este estudo caracteriza-se quanto à sua natureza como pesquisa aplicada, tendo em vista que, segundo Thiollent (1985), a pesquisa aplicada concentra-se em torno dos problemas presentes nas atividades das instituições, organizações, grupos ou atores sociais e está empenhada na elaboração de diagnósticos, identificação de problemas e busca de soluções.

A abordagem utilizada é qualitativa, pois busca compreender as necessidades e comportamentos dos usuários potenciais do sistema, bem como analisar a viabilidade de implementação em diferentes contextos institucionais.

\subsection{DELIMITAÇÃO DO ESTUDO}

A área de estudo deste projeto foi desenvolver um sistema web de compartilhamento de caronas especificamente voltado para o ambiente acadêmico, com foco na integração com portais institucionais de faculdades e universidades.

O desenvolvimento do protótipo foi realizado utilizando tecnologias web modernas e metodologias ágeis, permitindo iterações rápidas de desenvolvimento, teste e avaliação. A validação do conceito foi feita através de análise de requisitos, prototipagem e avaliação da viabilidade técnica e operacional.

\subsection{INSTRUMENTOS DE COLETA DE DADOS}

Para a coleta de dados e validação das necessidades identificadas, foram utilizados os seguintes instrumentos:

\subsubsection{Entrevistas Semiestruturadas}

Foram realizadas entrevistas com estudantes universitários para compreender:
\begin{itemize}
    \item Dificuldades atuais de mobilidade
    \item Experiências prévias com compartilhamento de caronas
    \item Expectativas em relação a um sistema institucional
    \item Preocupações sobre segurança e privacidade
\end{itemize}

As conversas documentadas nas Figuras \ref{fig:PrintUsuariodacarona}, \ref{fig:PrintCaroneiro} e \ref{fig:PrintCaroneiro2} exemplificam esse processo de coleta de dados primários.

\subsubsection{Análise Documental}

Foram analisados:
\begin{itemize}
    \item Regulamentos de trânsito nos campus universitários
    \item Políticas de mobilidade institucional
    \item Dados sobre a frota de veículos e vagas de estacionamento
    \item Estatísticas de transporte público nas proximidades das instituições
\end{itemize}

\subsubsection{Pesquisa Bibliográfica}

Realizada revisão sistemática da literatura sobre:
\begin{itemize}
    \item Sistemas de compartilhamento de caronas
    \item Economia colaborativa
    \item Mobilidade urbana sustentável
    \item Tecnologias web para sistemas colaborativos
    \item Segurança em aplicações web
\end{itemize}

\subsection{PROCEDIMENTOS DE ANÁLISE}

Os dados coletados foram analisados qualitativamente, identificando padrões nas necessidades relatadas e priorizando funcionalidades segundo critérios de:
\begin{itemize}
    \item Relevância para os usuários
    \item Viabilidade técnica
    \item Alinhamento com objetivos institucionais
    \item Contribuição para segurança e confiabilidade
\end{itemize}

A partir dessa análise, foram definidos os requisitos funcionais e não funcionais apresentados na Seção \ref{sec:requisitos}, bem como a arquitetura conceitual do sistema proposto.

\section{CONCLUSÃO}

Este trabalho apresentou o desenvolvimento de um sistema integrado de compartilhamento de caronas para o ambiente acadêmico, conectando estudantes motoristas com aqueles que necessitam de transporte dentro da comunidade universitária.

A análise de trabalhos relacionados demonstrou que, embora existam diversas plataformas de compartilhamento de caronas no mercado, como BlaBlaCar, Waze Carpool e iniciativas acadêmicas pontuais, nenhuma atende de forma completa as necessidades específicas do contexto universitário brasileiro. O sistema proposto apresenta diferenciais significativos através da integração institucional nativa, comunidade fechada e verificada, e recursos de segurança específicos para o ambiente acadêmico.

A definição detalhada dos requisitos funcionais e não funcionais permitiu estruturar uma solução tecnicamente viável que contempla aspectos essenciais como autenticação institucional, gestão de perfis, oferta e busca de caronas, sistema de avaliações e recursos de segurança. Os requisitos não funcionais estabelecem padrões de qualidade em termos de usabilidade, desempenho, segurança, confiabilidade e integração, garantindo que o sistema atenda aos critérios esperados para aplicações web modernas.

A metodologia de pesquisa aplicada, com abordagem qualitativa, possibilitou coletar dados primários através de entrevistas com estudantes, análise documental de políticas institucionais e revisão bibliográfica sistemática. Esses instrumentos de coleta forneceram subsídios fundamentais para compreender as necessidades reais dos usuários e definir funcionalidades alinhadas com o contexto acadêmico.

Os resultados obtidos indicam que a solução proposta pode contribuir significativamente para a redução dos problemas de mobilidade urbana no contexto universitário, promovendo sustentabilidade através da redução de veículos no campus, economia colaborativa mediante divisão de custos de transporte, e fortalecimento dos vínculos sociais na comunidade acadêmica.

O sistema representa uma inovação tecnológica que combina sustentabilidade ambiental, economia colaborativa e integração institucional, oferecendo uma alternativa viável e escalável aos problemas de mobilidade urbana enfrentados por milhões de estudantes universitários brasileiros. A arquitetura proposta permite replicação em diferentes instituições de ensino, potencializando o impacto social e ambiental da solução.

Como trabalhos futuros, sugere-se a implementação de um protótipo funcional para validação prática dos requisitos definidos, realização de testes com usuários reais, desenvolvimento de aplicativos móveis nativos para Android e iOS, e estabelecimento de parcerias com instituições de ensino para implantação piloto do sistema.

\postextual%% Elementos pós-textuais
\newpage
\printbibliography%% Referências

%% Fim do documento
\end{document}
